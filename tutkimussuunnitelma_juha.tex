\documentclass[12pt,a4paper,finnish,oneside]{article}

% Valitse 'input encoding':
%\usepackage[latin1]{inputenc} % merkistökoodaus, jos ISO-LATIN-1:tä.
\usepackage[utf8]{inputenc}   % merkistökoodaus, jos käytetään UTF8:a
% Valitse 'output/font encoding':
%\usepackage[T1]{fontenc}      % korjaa ääkkösten tavutusta, bittikarttana
\usepackage{ae,aecompl}       % ed. lis. vektorigrafiikkana bittikartan sijasta
% Kieli- ja tavutuspaketit:
\usepackage[finnish]{babel}
% Muita paketteja:
% \usepackage{amsmath}   % matematiikkaa
\usepackage{url}       % \url{...}

% Kappaleiden erottaminen ja sisennys
\parskip 1ex
\parindent 0pt
\evensidemargin 0mm
\oddsidemargin 0mm
\textwidth 159.2mm
\topmargin 0mm
\headheight 0mm
\headsep 0mm
\textheight 246.2mm

\pagestyle{plain}

% ---------------------------------------------------------------------

\begin{document}

% Otsikkotiedot: muokkaa tähän omat tietosi

\title{TIK.kand tutkimussuunnitelma:\\[5mm]
Juurisyyanalyysin soveltuvuus ohjelmistoprojektien retrospektiiveissä}

\author{Juha Viljanen, 80668R\\
Aalto-yliopisto,\\
\url{juha.o.viljanen@aalto.fi}}

\date{\today}

\maketitle

% ---------------------------------------------------------------------

\vspace{10mm}

% MUOKKAA TÄHÄN. Jos tarvitset tähän viitteitä, käytä
% tässä dokumentissa numeroviitejärjestelmää komennolla \cite{kahva}.
%
% Paljon kandidaatintöitä ohjanneen Vesa Hirvisalon tarjoama 
% sabluuna. Kursivoidut osat \emph{...} ovat kurssin henkilökunnan
% lisäämiä. 

\textbf{Kandidaatintyön nimi:} Juurisyyanalyysin soveltuvuus ohjelmistoprojektien retrospektiiveissä

\textbf{Työn tekijä:} Juha Viljanen

\textbf{Ohjaaja:} Timo Lehtinen


\section{Tiivistelmä tutkimuksesta}

Ketterässä ohjelmistokehityksessä kehitystiimi järjestää säännöllisesti iteraation päätteeksi retrospektiivejä. Näissä käydään läpi iteraation aikana löytyneitä ongelmia, sekä pohditaan tapoja ratkaista ne ja siten parantaa tiimin ohjelmistokehitysprosessia.Juurisyyanalyysi tarjoaa rakenteellisen tavan löytää ongelmien aiheuttajia ja voi siten auttaa ehkäisemään näiden ongelmien esiintymistä jatkossa. 

Tämä työ tutkii juurisyyanalyysin soveltuvuutta menetelmäksi ketterän ohjelmistokehitystiimin retrospektiiviin.

\section{Tavoitteet ja näkökulmat}

Tavoitteena on tutkia, minkälaisia menetelmiä aiemmassa kirjallisuudessa on esitetty juurisyyanalyysiä soveltaviin retrospektiiveihin? Lisäksi haluan selvittää, kokeeko ketterä ohjelmistokehitystiimi juurisyyanalyysin tehokkaaksi välineeksi retrospektiiviinsä.

\section{Tutkimusmateriaali}

Millaisen aineiston varaan perustat tutkimuksesi? Arvioi materiaalin
riittävyyttä asetettuihin tavoitteisiin nähden.

Pitää olla siis hieman kuvaa siitä, minkälaisen materiaalin kanssa
ollaan tekemisissä ja mitä sellaisen käsittelyyn tarvitaan (etenkin
siis tarvittavan ajan puolesta; ts. kuinka monta tuntia/minuuttia per
lähde?).

Juurisyyanalyysin käytöstä ongelmanratkaisumenetelmänä erilaisissa retrospektiiveissä löytyy materiaalia. Haasteena on ...

Rajaukset:
-haetaan materiaalia vain scopuksesta
-vain 1990 jälkeen tulleet paperit
-hakusanoina käytetään "retrospective", "postmortem analysis", "post-project review" ja "software engineering"

\section{Tutkimusmenetelmät}

Miten sen keräät materiaalisi tai saat sen käsiisi? Kuinka käsittelet
sen? Kuinka siitä tulee raportti?

Tavallaisesti kirjallisuustutkimuksen yhteydessä tämä on:
(a) lähderyhmien valinta,
(b) viitteiden ja lähteiden haku,
(c) lähteiden arviointi,
(d) lähteiden lukeminen,
(e) tiedon organisointi,
(f) raportointi.  % (f) tärkeää ettei jää vain lukemiseksi!

Kirjallisuustutkimuksen yleinen menetelmä pitää sovittaa tähän
nimenomaiseen aiheeseen sekä tekijän lähtökohtiin. Kuinka sinä teet
muistiinpanot (että myös kirjoitat etkä pelkästään lue). Eli tälle
pitää hieman miettiä omakohtaista vaiheistusta. Siis nähdä ihan
oikeasti, kuinka sinä saat tutkielman tehtyä.

Ja... raportointi ei ole kirjoittamista vaan jo kirjoitettujen
muistiinpanojen koostamista yhteneväksi teokseksi.

\subsection{Järjestelmällinen kirjallisuuskatsaus}
Suoritan kirjallisuustutkimuksen Kitchenhamin määrittelemänä järjestelmällisenä kirjallisuuskatsauksena. Suoritan haun pelkästään Scopus-tiedokannassa. Käyttämäni hakusanat tulevat olemaan "retrospective", "postmortem analysis", "post-project review" ja "software engineering". Näitä haetaan otsikosta ja avainsanoista. Artikkelien tulee olla julkaistu vuodesta 1990 alkaen. Vanhemmat artikkelit jäävät rajaukseni ulkopuolelle.

Löytyneistä luen abstraktin ja tarkistan sisältääkö artikkeli jonkin retrospektiivin (tai vastaavan) menetelmän. Nämä artikkelit kerään taulukkolaskentaohjelmaan tekemääni viitekokoelmaan, jossa on näiden artikkelien tekijät, otsikko, foorumi (lehden nimi, tms.), vuosi, Abstrakti ja merkintä siitä, onko artikkelissa mainittu menetelmä juurisyyanalyysi. Lisäksi merkitsen muistiin muut lähdeviitteeseen tarvittavat tiedot.

Tästä joukosta kandidaatin työhön päätyvät ne artikkelit, jotka sisältävät maininnan juurisyyanalyysistä. Tätä silmäilen artikkelia ja luen siitä abstraktin, sekä tarpeen mukaan tutkimusmenetelmästä, johtopäätökset ja johdantoa. Ne artikkelit, jotka käsittelevät juurisyyanalyysiä retrospektiivin menetelmänä luen tarkemmin. Näistä teen myös artikkelikohtaisia muistiinpanoja.

Systemaattisen kirjallisuustutkimuksen tarkoituksena on tehdä aineistonhakuprosessista tarkastettava ja toistettava.

%\citet[s. 21]{Kitchenham2010} on havainnut asian jos toisenkin. 
\subsection{Kenttätutkimus}
Kandidaatin työhön tekemäni kenttätutkimuksen suoritan ohjelmistoyrityksen ketterän kehitystiimin kanssa. Sovin ennalta tiimin kanssa asiasta ja sovimme tiimille sopivan ajan. Tiimi valmistelee yhden tai useamman aiheen (ongelman), joita he haluavat käsitellä juurisyyanalyysi-sessiossa. Minä ohjaan session. Käytämme työkaluna Aalto-yliopistolla kehitettyä ARCA-tool -web-sovellusta mahdollistamaan hajautetun tiimin yhteisen analyysin. Videoin session ruudunkaappaus-sovelluksella. 

Session jälkeen kerään osallistuneilta palautteen ennalta tekemäni palaute-lomakkeen avulla. Lisäksi kerään osallistujilta suullisen palautteen. Kysymykset käsittelevät tekemämme juurisyyanalyysin ja käyttämämme työkalun helppokäyttöisyyttä ja tehokkuutta. Kenttätutkimuksen analysointi tapahtuu tämän aineiston perusteella. Tutkimuksessa käytetty juurisyyanalyysimenetelmä on pelkistetty versio ARCA-menetelmästä.

\section{Haasteet}

Yleensä kaikkiin töihin liittyy kompastuskiviä. Ne on syytä tiedostaa
etukäteen. Yhdessä työssä aihe on suurpiirteinen (työn rajaaminen
vaikeaa), toisessa materiaalia on niukasti saatavissa, kolmannessa
taas materiaalia on hukkumiseen asti.  Eli, nämä pitäisi kyseisen
tutkimuksen osalta kirjata ylös, ja nähdä ne myös mahdollisuuksina
(positiivisina haasteina) ei ainostaan esteinä.

-materiaalin löytyminen SLR:n avulla.

\section{Resurssit}

Kandidaatintyön tekee Juha Viljanen ja sitä ohjaa Timo Lehtinen. Työhön tulen käyttämään suurinpiirtein kandidaatin työlle määritellyn tuntimäärän. Teen kandidaatintyötä muun opiskelun ja osa-aikatyön ohessa.
Työhön kuuluu myös kokeellinen osuus case-tutkimuksen muodossa ketterän ohjelmistokehitystiimin iteraation retrospektiivissä. Näitä sessioita tulee olemaan 1-2.

\section{Aikataulu}

\begin{tabular}{|p{30mm}|p{120mm}|}
\hline
\bf{Viikko} & \bf{Työn vaihe} \\ \hline
6   & Tiedonhaun tehtävä, tutkimussuunnitelma \\ \hline
7   & Lähteiden keräämistä, V1-palautus \\ \hline
8  & K2 tapaaminen, korjaukset V1:een ja tutkimussuunnitelmaan, lähteiden keräämistä, V2 kirjoittamista, mahdollinen toinen kenttätutkimus \\ \hline
9-10 &  Lähteiden keräämistä, V2 kirjoittaminen palautuskuntoon, Kielikeskuksen palautekeskustelu \\ \hline
11 &  K3 tapaaminen, korjaukset V2:een, V3:n kirjoittamista, opponointi , Kielikeskuksen palautekeskustelu \\ \hline
12-14 & V3:n kirjoittaminen palautuskuntoon, Kielikeskuksen palautekeskustelu \\ \hline
15 & K4 tapaaminen, korjaukset V3:een, V4:n kirjoittamista \\ \hline
16-17 & V4:n kirjoittaminen palautuskuntoon \\ \hline
18-19 & Kandiesityksen valmistelu, itse esitys, opponointi, opponointiraportin kirjoitus \\ \hline
20-21 & - \\ \hline
22 & Kypsyysnäyte \\ \hline
\end{tabular}


\section{Esittäminen}

\begin{enumerate}
	\item Johdanto
	\item Teoreettinen tausta
	\begin{enumerate}	
		\item Ketterä retrospektiivi
		\item Juurisyyanalyysi
	\end{enumerate}
	\item Menetelmät
	\begin{enumerate}
		\item Järjestemällinen kirjallisuuskatsaus
		\item  Kenttätutkimus
	\end{enumerate}
	\item Tulokset
	\begin{enumerate}
		\item  Kirjallisuuskatsaus
		\item Kenttätutkimus
	\end{enumerate}
	\item Pohdinta
	\item  Yhteenveto
\end{enumerate}


% -------------- Lähdeluettelo / reference list -----------------------
%
% Lähdeluettelo alkaa aina omalta sivultaan; pakota lähteet alkamaan
\clearpage
%
%
% Muista, että saat kirjallisuusluettelon vasta
%  kun olet kääntänyt ja kaulinnut "latex, bibtex, latex, latex"
%  (ellet käytä Makefilea ja "make")

% Viitetyylitiedosto aaltosci_t.bst; muokattu HY:n tktl-tyylistä.
\bibliographystyle{aaltosci_t}
% Katso myös tämän tiedoston yläosan "preamble" ja siellä \bibpunct.

% Muutetaan otsikko "Kirjallisuutta" -> "Lähteet"
\renewcommand{\refname}{Lähteet}  % article-tyyppisen
%\renewcommand{\bibname}{Lähteet}  % jos olisi book, report-tyyppinen

% Lisätään sisällysluetteloon
\addcontentsline{toc}{section}{\refname}  % article
%\addcontentsline{toc}{chapter}{\bibname}  % book, report

% Määritä kaikki bib-tiedostot
\bibliography{lahteet}
%\bibliography{thesis_sources,ietf_sources}

\label{pages:refs}
\clearpage         % erotetaan mahd. liitteet alkamaan uudelta sivulta



% ---------------------------------------------------------------------
%
% ÄLÄ MUUTA MITÄÄN TÄÄLTÄ LOPUSTA

% Tässä on käytetty siis numeroviittausjärjestelmää. 
% Toinen hyvin yleinen malli on nimi-vuosi-viittaus.

% \bibliographystyle{plainnat}
\bibliographystyle{finplain}  % suomi
%\bibliographystyle{plain}    % englanti
% Lisää mm. http://amath.colorado.edu/documentation/LaTeX/reference/faq/bibstyles.pdf

% Muutetaan otsikko "Kirjallisuutta" -> "Lähteet"
\renewcommand{\refname}{Lähteet}  % article-tyyppisen

% Määritä bib-tiedoston nimi tähän (eli lahteet.bib ilman bib)
\bibliography{lahteet}

% ---------------------------------------------------------------------

\end{document}
