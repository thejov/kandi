\documentclass[12pt,a4paper,finnish,oneside]{article}

% Valitse 'input encoding':
%\usepackage[latin1]{inputenc} % merkistökoodaus, jos ISO-LATIN-1:tä.
\usepackage[utf8]{inputenc}   % merkistökoodaus, jos käytetään UTF8:a
% Valitse 'output/font encoding':
%\usepackage[T1]{fontenc}      % korjaa ääkkösten tavutusta, bittikarttana
\usepackage{ae,aecompl}       % ed. lis. vektorigrafiikkana bittikartan sijasta
% Kieli- ja tavutuspaketit:
\usepackage[finnish]{babel}
% Muita paketteja:
% \usepackage{amsmath}   % matematiikkaa
\usepackage{url}       % \url{...}

% Kappaleiden erottaminen ja sisennys
\parskip 1ex
\parindent 0pt
\evensidemargin 0mm
\oddsidemargin 0mm
\textwidth 159.2mm
\topmargin 0mm
\headheight 0mm
\headsep 0mm
\textheight 246.2mm

\pagestyle{plain}

% ---------------------------------------------------------------------

\begin{document}

% Otsikkotiedot: muokkaa tähän omat tietosi

\title{TIK.kand tutkimussuunnitelma:\\[5mm]
Juurisyyanalyysin soveltuvuus ohjelmistoprojektien retrospektiiveissä}

\author{Juha Viljanen, 80668R\\
Aalto-yliopisto,\\
\url{juha.o.viljanen@aalto.fi}}

\date{\today}

\maketitle

% ---------------------------------------------------------------------

\vspace{10mm}

% MUOKKAA TÄHÄN. Jos tarvitset tähän viitteitä, käytä
% tässä dokumentissa numeroviitejärjestelmää komennolla \cite{kahva}.
%
% Paljon kandidaatintöitä ohjanneen Vesa Hirvisalon tarjoama 
% sabluuna. Kursivoidut osat \emph{...} ovat kurssin henkilökunnan
% lisäämiä. 

\textbf{Kandidaatintyön nimi:} Juurisyyanalyysin soveltuvuus ohjelmistoprojektien retrospektiiveissä

\textbf{Työn tekijä:} Juha Viljanen

\textbf{Ohjaaja:} Timo Lehtinen


\section{Tiivistelmä tutkimuksesta}

Ketterässä ohjelmistokehityksessä kehitystiimi järjestää säännöllisesti iteraation päätteeksi retrospektiivejä. Näissä käydään läpi iteraation aikana löytyneitä ongelmia, sekä pohditaan tapoja ratkaista ne ja siten parantaa tiimin ohjelmistokehitysprosessia. Juurisyyanalyysi tarjoaa rakenteellisen tavan löytää ongelmien aiheuttajia ja voi siten auttaa ehkäisemään näiden ongelmien esiintymistä jatkossa. 

Tämä työ tutkii juurisyyanalyysin soveltuvuutta menetelmäksi ketterän ohjelmistokehitystiimin retrospektiiviin.

\section{Tavoitteet ja näkökulmat}

Tavoitteena on tutkia, minkälaisia menetelmiä aiemmassa kirjallisuudessa on esitetty juurisyyanalyysiä soveltaviin retrospektiiveihin. Lisäksi haluan selvittää, kokeeko ketterä ohjelmistokehitystiimi juurisyyanalyysin tehokkaaksi välineeksi retrospektiiviinsä.

\section{Tutkimusmateriaali}

Juurisyyanalyysin käytöstä ongelmanratkaisumenetelmänä erilaisissa retrospektiiveissä löytyy materiaalia  arviolta 15-50 artikkelin verran. Kandidaatintyön ohjaajani on näyttänyt minulle löytämäänsä, sopivaa materiaalia (noin 15 artikkelia). Näitä en aio kuitenkaan ottaa suoraan käyttöön. Haasteena on nimittäin materiaalin löytäminen järjestelmällisellä, toistettavalla ja tieteellisellä menetelmällä. Haluan varmistua siitä, etten vahingossa ohita hyvää materiaalia sen takia, ettei siinä esimerkiksi mainittaisi termiä "root cause analysis".
Hakemani aineisto on rajattu artikkeleihin. Karkeamman rajauksen teen nopeammin (enintään 5 minuuttia per lähde), kun taas keskeiseltä vaikuttavaan materiaaliin voin käyttää enemmänkin aikaa (15 min - 60 min).

Kandidaationtyöni johdantoon tarvittavaa yleistietoa retrospektiiveistä ja juurisyyanalyysistä löytyy vielä runsaammin. Siinä haasteena voi olla minulle sopivan lähteen löytyminen. Retrospektiiveistä uskon sen löytyvän esimerkiksi ketterän kehityksen perusopuksista (kirjoista), ja juurisyyanalyysistä esimerkiksi sitä koskevien artikkelien taustatiedoista ja niiden lähteistä.

\section{Tutkimusmenetelmät}

\subsection{Järjestelmällinen kirjallisuuskatsaus}
Suoritan kirjallisuustutkimuksen Kitchenhamin määrittelemän järjestelmällisen kirjallisuuskatsauksen [1] muodossa. Suoritan haun pelkästään Scopus-tiedokannassa. Käyttämäni hakusanat tulevat olemaan "retrospective", "postmortem analysis", "post-project review" ja "software engineering". Näitä haetaan otsikosta ja avainsanoista. Artikkelien tulee olla julkaistu vuodesta 1990 alkaen. Vanhemmat artikkelit jäävät rajaukseni ulkopuolelle. Mikäli näillä rajauksilla löytyy liikaa artikkeleita, tarkennan hakua. Pyrin kuitenkin välttämään liian rajaavia termejä, kuten "root cause analysis" tai "agile". Näitä käyttämällä saattaisin ohittaa minua kiinnostavia artikkeleita, jotka puhuvat samoista asioista, mutta eri termejä käyttäen. Kirjaan eri hakusanayhdistelmillä löytyneiden tulosten määrän ylös.

Kun olen löytänyt Scopus-haullani hallittavan kokoisen otoksen (esim. noin 200 artikkelia), en enää vaihda hakusanaa. Rajaan näistä vielä ne pois, jotka otsikon ja hakutuloksen lyhyen kuvauksen perusteella vaikuttavat ehdottoman epäolennaisilta. Muista artikkeleista luen abstraktin ja tarkistan sisältääkö artikkeli jonkin retrospektiivissä (tai vastaavassa) käytettävän menetelmän kuvauksen. Nämä artikkelit kerään taulukkolaskentaohjelmaan tekemääni viitekokoelmaan, jossa on näiden artikkelien tekijät, otsikko, foorumi (lehden nimi, tms.), vuosi, abstrakti ja merkintä siitä, onko artikkelissa kuvattu retrospektiivin menetelmä juurisyyanalyysi. Lisäksi merkitsen muistiin muut lähdeviitteeseen tarvittavat tiedot.

Tästä joukosta kandidaatin työhön päätyvät ne artikkelit, jotka sisältävät maininnan juurisyyanalyysistä retrospektiivin menetelmänä. Ne artikkelit silmäilen läpi ja luen siitä abstraktin, sekä tarpeen mukaan tutkimusmenetelmästä, johtopäätökset ja johdantoa. Näistä teen myös artikkelikohtaisia muistiinpanoja, jotka liitän taulukkoon omalle sarakkeelleen. Näitä muistiinpanoja käytän varsinaista kandidaatintyötä kirjoittaessani.

Systemaattisen kirjallisuustutkimuksen tarkoituksena on tehdä aineistonhakuprosessista tarkastettava ja toistettava.

\subsection{Kenttätutkimus}
Kandidaatin työhön tekemäni kenttätutkimuksen suoritan ohjelmistoyrityksen ketterän kehitystiimin kanssa. Sovin ennalta tiimin kanssa asiasta ja sovimme tiimille sopivan ajan. Tiimi valmistelee yhden tai useamman aiheen (ongelman), joita he haluavat käsitellä juurisyyanalyysi-sessiossa. Minä ohjaan session. Käytämme työkaluna Aalto-yliopistolla kehitettyä ARCA-tool -web-sovellusta mahdollistamaan hajautetun tiimin yhteisen analyysin. Videoin session ruudunkaappaus-sovelluksella. 

Session jälkeen kerään osallistuneilta palautteen ennalta tekemäni palaute-lomakkeen avulla. Lisäksi kerään osallistujilta suullisen palautteen. Kysymykset käsittelevät tekemämme juurisyyanalyysin ja käyttämämme työkalun helppokäyttöisyyttä ja tehokkuutta. Kenttätutkimuksen analysointi tapahtuu tämän aineiston perusteella. Tutkimuksessa käytetty juurisyyanalyysimenetelmä on pelkistetty versio ARCA-menetelmästä.

\section{Haasteet}

Tällä hetkellä tuntuu siltä, että kandidaatintyöni suurin haaste tulee olemaan materiaalin hakuprosessi järjestelmällisen kirjallisuuskatsauksen menetelmää käyttäen. Menetelmä on minulle vielä vieras ja vaikka ymmärrän sen tuoman arvon, minusta tuntuu myös siltä, että se vaikeuttaa lähteiden hakuprosessiani. Toisaalta selätettyäni tämän haasteen tiedän oppineeni tieteellisen tavan löytää sopivaa kirjallisuutta.

\section{Resurssit}

Kandidaatintyön tekee Juha Viljanen ja sitä ohjaa Timo Lehtinen. Työhön tulen käyttämään suurinpiirtein kandidaatin työlle määritellyn tuntimäärän. Teen kandidaatintyötä muun opiskelun ja osa-aikatyön ohessa.
Työhön kuuluu myös kokeellinen osuus case-tutkimuksen muodossa ketterän ohjelmistokehitystiimin iteraation retrospektiivissä. Näitä sessioita tulee olemaan 1-2.

\clearpage
\section{Aikataulu}

\begin{tabular}{|p{30mm}|p{120mm}|}
\hline
\bf{Viikko} & \bf{Työn vaihe} \\ \hline
6   & Tiedonhaun tehtävä, tutkimussuunnitelma \\ \hline
7   & Lähteiden keräämistä, V1-palautus \\ \hline
8  & K2 tapaaminen, korjaukset V1:een ja tutkimussuunnitelmaan, lähteiden keräämistä, V2 kirjoittamista, mahdollinen toinen kenttätutkimus \\ \hline
9-10 &  Lähteiden keräämistä, V2 kirjoittaminen palautuskuntoon, Kielikeskuksen palautekeskustelu \\ \hline
11 &  K3 tapaaminen, korjaukset V2:een, V3:n kirjoittamista, opponointi , Kielikeskuksen palautekeskustelu \\ \hline
12-14 & V3:n kirjoittaminen palautuskuntoon, Kielikeskuksen palautekeskustelu \\ \hline
15 & K4 tapaaminen, korjaukset V3:een, V4:n kirjoittamista \\ \hline
16-17 & V4:n kirjoittaminen palautuskuntoon \\ \hline
18-19 & Kandiesityksen valmistelu, itse esitys, opponointi, opponointiraportin kirjoitus \\ \hline
20-21 & - \\ \hline
22 & Kypsyysnäyte \\ \hline
\end{tabular}

\section{Esittäminen}

\begin{enumerate}
	\item Johdanto
	\item Teoreettinen tausta
	\begin{enumerate}	
		\item Ketterä retrospektiivi
		\item Juurisyyanalyysi
	\end{enumerate}
	\item Menetelmät
	\begin{enumerate}
		\item Järjestemällinen kirjallisuuskatsaus
		\item  Kenttätutkimus
	\end{enumerate}
	\item Tulokset
	\begin{enumerate}
		\item  Kirjallisuuskatsaus
		\item Kenttätutkimus
	\end{enumerate}
	\item Pohdinta
	\item  Yhteenveto
\end{enumerate}

\clearpage
\section{Lähteet}
1. Kitchenham B. et al. Systematic literature reviews in software engineering - A tertiary study. Information and Software Technology, 2010. Vol. 52:8. S. 792-805.

% ---------------------------------------------------------------------
%
% ÄLÄ MUUTA MITÄÄN TÄÄLTÄ LOPUSTA

% Tässä on käytetty siis numeroviittausjärjestelmää. 
% Toinen hyvin yleinen malli on nimi-vuosi-viittaus.

% \bibliographystyle{plainnat}
\bibliographystyle{finplain}  % suomi
%\bibliographystyle{plain}    % englanti
% Lisää mm. http://amath.colorado.edu/documentation/LaTeX/reference/faq/bibstyles.pdf

% Muutetaan otsikko "Kirjallisuutta" -> "Lähteet"
\renewcommand{\refname}{Lähteet}  % article-tyyppisen

% Määritä bib-tiedoston nimi tähän (eli lahteet.bib ilman bib)
\bibliography{lahteet}

% ---------------------------------------------------------------------

\end{document}
