% --------------------------------------------------------------------

\section{Johdanto}

Ohjelmistoprojekteissa tulee väistämättä vastaan ongelmia. Näiden järjestelmällinen analysointi ja ehkäiseminen jatkuvana osana kehitysprosessia parantaa ohjelmistoprojektin mahdollisuuksia onnistua. Ketterässä ohjelmistokehityksessä kehitystiimi järjestää säännöllisesti iteraation päätteeksi retrospektiivejä. Niissä käydään läpi iteraation aikana löytyneitä ongelmia, sekä pohditaan tapoja ratkaista ne ja siten parantaa tiimin ohjelmistokehitysprosessia. Juurisyyanalyysi tarjoaa rakenteellisen tavan löytää ongelmien aiheuttajia ja voi siten auttaa ehkäisemään näiden ongelmien esiintymistä jatkossa.

Tämä kandidaatintyö käsittelee sitä, miten juurisyyanalyysi soveltuu menetelmäksi ketterän ohjelmistokehitystiimin retrospektiiviin.

Iteraation lopussa pidettävä retrospektiivi on olennainen osa ketterän ohjelmistokehitysprosessin runkoa. Vaikka sen tavoitteet on yleensä määritelty tarkasti kunkin metodologian kuvauksessa, on sen toteutustapa jätetty yleensä tiimin päätettäväksi. Esimerkiksi Scrum-metodologian kuvauksessa on kuvattu retrospektiivien tavoitteet, muttei niiden saavuttamiseen johtavia menetelmiä \citep{ScrumGuide2011}. Tämän kandidaatintyön tarkoituksena selvittää juurysyyanalyysin soveltuvuutta ketterän retrospektiivin menetelmäksi. Olennaista on se, minkälainen juurisyyanalyysi-menetelmä siihen soveltuu. Menetelmän tulee olla erittäin kevyt ja yksinkertainen, jotta sen käyttöönotto lyhyehköissä iteraatio-retrospektiiveissä olisi mielekästä.

Kandidaatintyön tutkimuskysymykset ovat seuraavat:
\begin{enumerate}
\item Minkälaisia menetelmiä aiemmassa kirjallisuudessa on esitetty juurisyyanalyysiä soveltaviin retrospektiiveihin?
\item Kokevatko ketterät ohjelmistokehitystiimit juurisyyanalyysiä soveltavan retrospektiivin tehokkaaksi?
\end{enumerate}

Kandidaatintyön tavoitteena on selvittää järjestelmällisen kirjallisuuskatsauksen \citep{Kitchenham2010} muodossa vastaus ensimmäiseen tutkimuskysymykseen. Toiseen tutkimuskysymykseen haetaan vastausta kenttätestillä.

Kirjallisuustutkimuksen työmäärän pitämiseksi järkevänä, aineistohaut rajoitetaan Scopus-tiedokannan tieteellisiin artikkeleihin. Käytettävät hakusanat, joita haetaan artikkelien otsikosta ja avainsanoista, ovat "retrospective", "postmortem analysis", "post-project review" ja "software engineering". Artikkelien tulee olla julkaistu aikaisintaan vuonna 1990. Mikäli näillä rajauksilla löytyy liikaa artikkeleita, tarkennetaan hakua. Liian rajaavien termien, kuten "root cause analysis" tai "agile", käyttöä pyritään kuitenkin välttämään. Näitä käyttämällä saattaisi ohittaa kiinnostavia artikkeleita, jotka käsittelevät olennaisia asioita, mutta eri termejä käyttäen. Eri hakusanayhdistelmillä löytyneiden tulosten määrä kirjataan ylös.

Hakutuloksista olennaisia ovat sellaiset, jotka kuvaavat jonkinlaista menetelmää käytettäväksi retrospektiiveihin. Nämä artikkelit kerätään taulukkolaskentaohjelman viitekokoelmaan, jossa ylläpidetään artikkeleista kaikkia lähdeviitteeseen tarvittavia tietoja, sekä merkinnän siitä, onko artikkelissa kuvattu retrospektiivin menetelmä juurisyyanalyysi. Ne artikkelit, jossa menetelmä sisältää juurisyyanalyysin, päätyvät kandidaatin työhön. Systemaattisen kirjallisuustutkimuksen tarkoituksena on tehdä aineistonhakuprosessista tarkastettava ja toistettava.

Kandidaatintyössä tehdään synteesi kirjallisuuskatsauksessa kerätyssä aineistossa esitetyistä retrospektiivien juurisyyanalyysi-menetelmistä. On mahdollista, että osa aineistosta käsittelee raskaampaa, suuremman mittakaavan retrospektiiviä, joka pidettäisiin esimerkiksi projektin jälkeieen (post project review). Mikäli synteesin kuvaama menetelmä osoittautuu ketterän ohjelmistokehitystiimin retrospektiiviin liian raskaaksi, karsitaan siitä tähän tarkoitukseen ylimitoitetut kohdat pois. Lopputuloksen tulisi olla sellainen, että tiimi voi suorittaa sen ketterälle retrospektiiville varatussa, verrattain lyhyessä ajassa.

Kandidaatin työn kenttätutkimus suoritetaan ohjelmistoyrityksen ketterän kehitystiimin kanssa. Juurisyyanalyysi-sessiosta sovitaan tiimin kanssa etukäteen. Tiimi valmistelee yhden tai useamman aiheen (ongelman), joita he haluavat käsitellä juurisyyanalyysi-sessiossa. Kandidaatintyön tekijä on ohjaamassa sessiota ja työn valvoja seuraa tapahtumia sivusta. Työkaluna käytetään Aalto-yliopistolla kehitettyä ARCA-tool -web-sovellusta mahdollistamaan hajautetun tiimin yhteisen analyysin. Sessio videoidaan ruudunkaappaus-sovelluksella. 

Juurisyyanalyysi-session jälkeen osallistuneilta kerätään palaute ennalta tehdyn palaute-lomakkeen avulla. Lisäksi kerätään suullinen palautte. Kysymykset käsittelevät tiiimin suorittaman juurisyyanalyysin ja siihen käytetyn työkalun helppokäyttöisyyttä ja tehokkuutta. Kenttätutkimuksen analysointi tapahtuu tämän kerätyn aineiston perusteella. Tutkimuksessa käytetty juurisyyanalyysimenetelmä on pelkistetty versio ARCA-menetelmästä \citep{Lehtinen2011}.

Kandidaatintyössä esitellään ensin lyhyesti työn ymmärtämiseen tarvittava teoreettinen tausta, eli määrittellään termit "ketterä retrospektiivi" ja "juurisyyanalyysi". Tämän jälkeen kuvataan työssä käytettyjä menetelmiä, eli järjestelmällistä kirjallisuuskatsausta ja kenttätutkimuksen menetelmiä. Sitten tuodaan julki näillä menetelmillä saadut tulokset ja tehdään näistä tarvittavat johtopäätökset. Kandidaatintyön viimeinen kappale on yhteenveto.

\section{Teoreettinen tausta}
\subsection{Ketterä retrospektiivi}
Ketterissä ohjelmistokehitysmenetelmissä on määritelty, että on säännöllisin väliajoin pidetetävä reflektointi, jossa tiimi pohtii tapoja tulla tehokkaammaksi. Nämä kehitettyjen parannusten perusteella tiimi muuttaa toimintaansa \citep{AgileManifestoPrinciples}. Kevyt retrospektiivi sessio on yksi tapa toteuttaa tätä periaatetta. Projektin lopputuloksen kannalta retrospektiivejä on järkevää pitää lyhyin väliajoin. Tällöin tiimin kohtaamat ongelmat ja niihin liittyvät yksityiskohdat ovat tuoreessa muistissa ja parannusehdotukset voidaan ottaa suoraan käyttöön parantaen siten projektin lopputuloksen laatua. \citep{Cockburn2002}

\subsection{Juurisyyanalyysi}
Juurisyyanalyysi on rakenteellinen tapa tutkia ongelmia ja tunnistaa niiden aiheuttajia. Ideana on se, että korjaamalla syyn aiheuttavia ongelmia voidaan ehkäistä saman ongelman syntymistä uudestaan -- tai ainakin vähentää ongelman uudelleenesiintymisen todennäköisyyttä. \citep{Lehtinen2011} Juurisyyanalyysin avulla voidaan tutkia ongelmien lisäksi myös onnistumisten syitä. \citep{Bjornson2009} Juurisyylle on useita määritelmiä. Se voi tarkoittaa mitä tahansa ongelman aiheuttavaa syytä, syyketjun perimmäisintä syytä tai syyksi, johon johtoporras voi vaikuttaa. Juurisyyanalyysin tuloksia voidaan käyttää apuna prosessinkehityksessä. \citep{Lehtinen2011}

\section{Tutkimusmenetelmät}
\subsection{Systemaattinen kirjallisuuskatsaus}
Systemaattisessa kirjallisuuskatsauksessa (SLR, Systematic Literature Review) tehdään valitusta aiheesta kattava arviointi, jossa käytetään luotettavaa, tarkkaa ja toistettavisaa olevaa menetelmää. SLR on kirjallisuuskatsauksen muoto, eli siinä käydään läpi aiempia tutkimuksia, jotka ovat olennaisia omien tutkimuskysymysten valossa. Kerätyn kirjallisuuden pohjalta muodostetaan synteesi. \citep{Kitchenham2007}

SLR:n erityispiirteenä on se, että kirjallisuuden etsiminen, valikointi ja valitun kirjallisuuden analysointi pyritään tekemään toistettavasti ja puolueettomasti. Kitchenham perustelee SLR-menetelmää toteamalla, että kirjallisuuskatsaus, joka ei ole SLR:n tapaan perusteellinen ja tasapuolinen ei tarjoa paljoa tieteellistä arvoa. Sen tulokset ovat todennäköisemmin puolueetetomia, eli tutkijan kannasta riippumattomia. \citep{Kitchenham2007}

Systemaattinen kirjallisuuskatsaus koostuu kolmesta päävaiheesta: suunnittelu-, toteutus- ja raportointivaiheista. Suunnitteluvaiheessa tunnistetaan kirjallisuuskatsauksen tarve, määritellään tutkimuskysymykset, sekä protokolla katsauksen suorittamiselle. 

Toteutusvaiheessa tehdään aineistohakuja ja valitaan ennaltamääritellyin kriteerein tutkimukselle oleelliset artikkelit. Artikkelien laatua ja sitä kautta luotettavuutta arvioidaan. Tehdyistä hauista kirjataan kaikki kirjallisuuskatsauksen toistamiseen ja sen laadun arviointiin tarvittava tieto, kuten haussa käytetyt hakukoneet, hakusanat, löydettyjen tulosten määrä ja jopa tuloslista. Valitusta aineistosta kerätään tietoa talteen ja sen merkittävyyttä omalle tutkimukselle arvioidaan ennalta määritellyin kriteerein. Kerätyn ja merkittäväksi valitun tiedon perusteella muodostetaan synteesi. Raportointivaiheessa kirjoitetaan tulokset ylös ja arvioidaan tuloksia.

Tähän kandidaatintyöhön valittiin Kitchenhamin määrittelemä systemaattinen kirjallisuuskatsaus, jotta kerätty aineistolista ja sen pohjalta tehty synteesi olisi tieteellisesti merkittävämpää, sekä mahdollisesti myös muulle tutkimukselle käyttökelpoista.

\subsection{Kenttätutkimus}
Kenttätutkimus suoritetaan keskikokoisessa ohjelmistoyrityksessä, jonka ketterälle ohjelmistokehitystiimille kandidaatintyön kirjoittaja pitää retrospektiivin. Kyseessä on noin kymmenen jäsenen kokoinen eri maiden välillä toimiva hajautettu tiimi. Retrospektiivissä tiimi pohtii edellisessä sprintissä haasteita aiheuttaneita ongelmia höyhenenkevyttä juurisyyanalyysi-menetelmää käyttäen. Retrospektiivin kommunikaatio tapahtuu videoneuvottelutyökalua käyttäen. Juurisyyanalyysissä käytetään Aalto-yliopistolla kehitettyä ARCA-tool -juurysyyanalyysityökalun eri versioita. ARCA-tool on Internetin ylitse toimiva juurisyyanalyysityökalu, joka mahdollistaa kollaboratiivisen ja reaaliaikaisen juurisyyanalyysin tekemisen. Jokainen voi osallistua analyysin tekemiseen omalta työpisteeltään pelkää Internet-selainta käyttäen.

Itse juurisyyanalyysimenetelmän on täytettävä tietyt kriteerit, jotta se soveltuisi tiimin retrospektiiviin. Ensinnäkin tiimi on asettanut retrospektiivin aikaikkunaksi yhden tunnin. Tiimin pitää siis saada valittua juurisyyanalyysimenetelmää käyttämällä konkreettisia tuloksia tuon tunnin jälkeen. Tavoitteena on ymmärtää tiimin valitsemat ongelmat ja niiden aiheuttajat syvällisemmin, sekä muodostaa konkreettisia toimia tiimin seuraavan sprintin backlogille näiden ongelmien ehkäisemiseksi. Lisäksi menetelmän pitää olla helposti omaksuttava siten, ettei retrospektiiviin osallistuville tarvitse pitää erillistä koulutusta menetelmästä, vaan se voidaan kouluttaa heille muutamalla lauseella ennen aloittamista.

Juurisyyanalyysi-sessiot taltioidaan ruudunkaappaus-sovelluksella, jolloin sisältyy sekä ARCA-tool:in näkymä koko analyysin ajan, sekä kommunikointiin käytetty videoneuvottelukuva. Taltioinnin tarkoituksena on se, ettei kandidaatintyön kirjoittalta jää huomaamatta yksityiskohtia session kulusta, osallistujien tunnetiloista ja reaktioista tai syyverkon muodostumisesta.

Sessioiden jälkeen kandidaatintyön tekijä kerää osallistujilta palautteen, joka koostuu sekä kyselylomakkeesta, että vapaammasta suullisesta tai pikaviestimen avulla kerätystä palautteesta. Molemmissa esitetyt kysymykset käsittelevät sitä, miten tehokkaaksi ja helppokäyttöiseksi osallistujat kokivat käytetyn juurisyyanalyysimenetelmän ja ARCA-tool:in verrattuna aiempiin prosessinkehitysmenetelmiinsä. Kyselylomake on Google Forms -työkalulla tehty Internet-lomake, jossa on yhteensä 26 kysymystä. 

\subsubsection{Ensimmäinen case}
Ensimmäisen retrospektiivin juurisyyanalyysimenetelmäksi on valittu ARCA-menetelmään perustuva, erittäin yksinkertainen ja pelkistetty,  höyhenenkevyt menetelmä, jonka kuvaan seuraavaksi.

Retrospektiivin lähtöasetelmana on se, että kullakin tiimin jäsenellä on Internet-selaimessaan auki retrospektiiviä varten luotu uusi analyysi ARCA-tool:issa. Tiimi on etukäteen valinnut ongelmat, joihin analyysissä halutaan pureutua.

Fasilitaattorin ohjauksessa tiimin jäsenet keräävät ongelmiin johtaneita syitä. Syiden ja niiden alisyiden kerääminen tehdään kahdessa iteraatioissa. Näissä viiden minuutin pituisissa iteraatioissa tiimin jäsenet lisäävät omatoimisesti ongelmiin ja niiden syihin johtaneita syitä ARCA-tool:issa näkyvään syyverkkoon. Kunkin iteraation jälkeen lisätyt syyt käydään läpi, jotta kaikki saavat yleiskuvan siitä, minkälaisista syistä syyverkko koostuu. Jokainen tiimin jäsen esittelee lisäämänsä syyt. Mikäli läpikäynnin yhteydessä esiintyy uusia syitä, ne lisätään verkkoon. Toisen iteraation jälkeen tiimin scrum master summaa koko syyverkon lyhyesti. 

Viimeinen askel on juurisyiden löytäminen. Fasilitaattori korostaa sitä, että juurisyyt, joita retrospektiivissä haetaan ovat ne syvimmät syyt, joihin tiimi voi vaikuttaa. Mielellään ne ovat sellaisia, joihin vaikuttamisen voi siirtää seuraavan sprintin backlogille. Juurisyiden löytäminen tapahtuu siten, että jokainen tiimin jäsen saa ehdottaa yhtä syytä "tykkäämällä" siitä ARCA-tool:issa. Ne 1-3 syytä, jotka ovat saaneet eniten tykkäyksiä, valitaan juurisyiksi.

Ensimmäisessä case-retrospektiivissä käytetään ARCA-tool:in ensimmäistä versiota, joka kehitettiin 2011-2012 lukuvuonna Aalto-yliopiston Ohjelmistokehitysprojekti-kurssilla (T-76.4115). Siinä on kaikki olennaisimmat juurisyyanalyysin tekemiseen tarvittavat toiminnot, jotka sisältävät muun muassa syyverkon kollaboratiivisen rakentamisen, tiettyjen syiden painottamisen tykkäys-toiminnolla, sekä korjaavien ideoiden lisäämisen.

\subsubsection{Toinen case}
Mikäli toinen RCA-sessio saadaan järjestettyä case-yrityksen tiimin kanssa, käytetään siinä eri juurisyyanalyysimenetelmää ja eri versiota ARCA-tool:ista. Menetelmäksi valitaan systemaattisen kirjallisuuskatsauksen tuloksena syntynyttä synteesiä. Voi olla, että synteesiin täytyy tehdä muutoksia, jotta se sopii tiimin retrospektiiville asettamiin raameihin. Retrospektiivissä käytetään ARCA-toolin uusinta versiota, joka kehitettiin 2012-2013 lukuvuonna Aalto-yliopiston Ohjelmistokehitysprojekti-kurssilla. Siinä on parannettu edellisen version käytettävyyttä, sekä lisätty uutena ominaisuutena mahdollisuus saada tehdystä analyysistä yksinkertaistetun yleiskuvan tarjoava näkymä.

\section{Tulokset}
\subsection{Systemaattinen kirjallisuuskatsaus}
\subsection{Kenttätutkimus}


\section{Pohdinta}

\section{Yhteenveto}
