% --------------------------------------------------------------------

\section{Johdanto}

Ohjelmistoprojekteissa tulee väistämättä vastaan ongelmia. Näiden järjestelmällinen analysointi ja ehkäiseminen jatkuvana osana kehitysprosessia parantaa ohjelmistoprojektin mahdollisuuksia onnistua. Ketterässä ohjelmistokehityksessä kehitystiimi järjestää säännöllisesti iteraation päätteeksi retrospektiivejä. Niissä käydään läpi iteraation aikana löytyneitä ongelmia, sekä pohditaan tapoja ratkaista ne ja siten parantaa tiimin ohjelmistokehitysprosessia. Juurisyyanalyysi tarjoaa rakenteellisen tavan löytää ongelmien aiheuttajia ja voi siten auttaa ehkäisemään näiden ongelmien esiintymistä jatkossa.

Tämä kandidaatintyö käsittelee sitä, miten juurisyyanalyysi soveltuu menetelmäksi ketterän ohjelmistokehitystiimin retrospektiiviin.

Iteraation lopussa pidettävä retrospektiivi on olennainen osa ketterän ohjelmistokehitysprosessin runkoa. Vaikka sen tavoitteet on yleensä määritelty tarkasti kunkin metodologian kuvauksessa, on sen toteutustapa jätetty yleensä tiimin päätettäväksi. Esimerkiksi Scrum-metodologian kuvauksessa on kuvattu retrospektiivien tavoitteet, muttei niiden saavuttamiseen johtavia menetelmiä \citep{ScrumGuide2011}. Tämän kandidaatintyön tarkoituksena selvittää juurysyyanalyysin soveltuvuutta ketterän retrospektiivin menetelmäksi. Olennaista on se, minkälainen juurisyyanalyysi-menetelmä siihen soveltuu. Menetelmän tulee olla erittäin kevyt ja yksinkertainen, jotta sen käyttöönotto lyhyehköissä iteraatio-retrospektiiveissä olisi mielekästä.

Kandidaatintyön tutkimuskysymykset ovat seuraavat:
\begin{enumerate}
\item Minkälaisia menetelmiä aiemmassa kirjallisuudessa on esitetty juurisyyanalyysiä soveltaviin retrospektiiveihin?
\end{enumerate}

Kandidaatintyön tavoitteena on selvittää järjestelmällisen kirjallisuuskatsauksen \citep{Kitchenham2010} muodossa vastaus tutkimuskysymykseen.

Kirjallisuustutkimuksen työmäärän pitämiseksi järkevänä, aineistohaut rajoitetaan Scopus-tietokannan tieteellisiin artikkeleihin. Käytettävät hakusanat, joita haetaan artikkelien otsikosta ja avainsanoista, ovat "retrospective", "postmortem analysis", "post-project review" ja "software engineering". Artikkelien tulee olla julkaistu aikaisintaan vuonna 1990. Mikäli näillä rajauksilla löytyy liikaa artikkeleita, tarkennetaan hakua. Liian rajaavien termien, kuten "root cause analysis" tai "agile", käyttöä pyritään kuitenkin välttämään. Näitä käyttämällä saattaisi ohittaa kiinnostavia artikkeleita, jotka käsittelevät olennaisia asioita, mutta eri termejä käyttäen. Eri hakusanayhdistelmillä löytyneiden tulosten määrä kirjataan ylös.

Hakutuloksista olennaisia ovat sellaiset, jotka kuvaavat jonkinlaista menetelmää käytettäväksi retrospektiiveihin. Nämä artikkelit kerätään taulukkolaskentaohjelman viitekokoelmaan, jossa ylläpidetään artikkeleista kaikkia lähdeviitteeseen tarvittavia tietoja, sekä merkinnän siitä, onko artikkelissa kuvattu retrospektiivin menetelmä juurisyyanalyysi. Ne artikkelit, jossa menetelmä sisältää juurisyyanalyysin, päätyvät kandidaatin työhön. Systemaattisen kirjallisuustutkimuksen tarkoituksena on tehdä aineistonhakuprosessista tarkastettava ja toistettava \citep{Kitchenham2007}.

Kandidaatintyössä tehdään synteesi kirjallisuuskatsauksessa kerätyssä aineistossa esitetyistä retrospektiivien juurisyyanalyysi-menetelmistä. On mahdollista, että osa aineistosta käsittelee raskaampaa, suuremman mittakaavan retrospektiiviä, joka pidettäisiin esimerkiksi projektin jälkeieen (post project review). Mikäli synteesin kuvaama menetelmä osoittautuu ketterän ohjelmistokehitystiimin retrospektiiviin liian raskaaksi, karsitaan siitä tähän tarkoitukseen ylimitoitetut kohdat pois. Lopputuloksen tulisi olla sellainen, että tiimi voi suorittaa sen ketterälle retrospektiiville varatussa, verrattain lyhyessä ajassa.

Kandidaatintyössä esitellään ensin lyhyesti työn ymmärtämiseen tarvittava teoreettinen tausta, eli määrittellään termit "ketterä retrospektiivi" ja "juurisyyanalyysi". Tämän jälkeen kuvataan työssä käytettyjä menetelmiä, eli järjestelmällistä kirjallisuuskatsausta. Sitten tuodaan julki näillä menetelmillä saadut tulokset ja tehdään näistä tarvittavat johtopäätökset. Kandidaatintyön viimeinen kappale on yhteenveto.

\section{Teoreettinen tausta}
\subsection{Ketterä retrospektiivi}
Ketterissä ohjelmistokehitysmenetelmissä on määritelty, että säännöllisin väliajoin pidetään reflektointi, jossa tiimi pohtii tapoja tulla tehokkaammaksi. Nämä kehitettyjen parannusten perusteella tiimi muuttaa toimintaansa \citep{AgileManifestoPrinciples}. Kevyt retrospektiivisessio on tapa toteuttaa tätä periaatetta. Projektin lopputuloksen kannalta retrospektiivejä on järkevää pitää lyhyin väliajoin. Tällöin tiimin kohtaamat ongelmat ja niihin liittyvät yksityiskohdat ovat tuoreessa muistissa ja parannusehdotukset voidaan ottaa suoraan käyttöön ja siten pyrkiä parantamaan projektin lopputuloksen laatua. \citep{Cockburn2002}

XP- ja Scrum-metodologiassa retrospektiivejä suositellaan pidettäväksi joka iteraation päätteeks. Niissä tiimi reflektoi sitä, missä he ovat viime iteraation aikana onnistuneet hyvin ja missä on vielä kehitettävää. \citep{Lindstrom2004, ScrumGuide2011} Retrospektiivin avulla tiimi saa palautetta työstään. Retrossa voidaan muun muassa pohtia sitä, miten kurinalaisesti tiimi on noudattanut seuraamansa metodologian käytäntöjä ja voisiko niitä räätälöidä sopimaan paremmin tiimin tarpeisiin. \citep{Lindstrom2004}

\subsection{Juurisyyanalyysi}
Juurisyyanalyysi on rakenteellinen tapa tutkia ongelmia ja tunnistaa niiden aiheuttajia. Ideana on se, että korjaamalla syyn aiheuttavia ongelmia voidaan ehkäistä saman ongelman syntymistä uudestaan -- tai ainakin vähentää ongelman uudelleenesiintymisen todennäköisyyttä. \citep{Lehtinen2011} Juurisyyanalyysin avulla voidaan tutkia ongelmien lisäksi myös onnistumisten syitä. \citep{Bjornson2009} Juurisyylle on useita määritelmiä. Se voi tarkoittaa mitä tahansa ongelman aiheuttavaa syytä, syyketjun perimmäisintä syytä tai syytä, johon johtoporras voi vaikuttaa. Juurisyyanalyysin tuloksia voidaan käyttää apuna prosessinkehityksessä. \citep{Lehtinen2011}

\section{Systemaattinen kirjallisuuskatsaus tutkimusmenetelmänä}
\subsection{Taustaa}
Systemaattisessa kirjallisuuskatsauksessa (SLR, Systematic Literature Review) tehdään kattava arviointi valitusta aiheesta. Arvioinnissa käytetään luotettavaa, tarkkaa ja toistettavisaa olevaa menetelmää. SLR on kirjallisuuskatsauksen muoto, mikä tarkoittaa sitä, että siinä käydään läpi aiempia tutkimuksia, jotka ovat olennaisia omien tutkimuskysymysten valossa. Kerätyn kirjallisuuden pohjalta muodostetaan synteesi.\citep{Kitchenham2007}

SLR:n erityispiirteenä on se, että kirjallisuuden etsiminen, valikointi ja valitun kirjallisuuden analysointi pyritään tekemään toistettavasti ja puolueettomasti. Kitchenham perustelee SLR-menetelmän hyötyä toteamalla, että kirjallisuuskatsaus, joka ei ole SLR:n tapaan perusteellinen ja tasapuolinen ei tarjoa paljoa tieteellistä arvoa. Sen tulokset ovat todennäköisemmin puolueettomia, eli tutkijan kannasta riippumattomia. \citep{Kitchenham2007}

Systemaattinen kirjallisuuskatsaus koostuu kolmesta päävaiheesta: suunnittelu-, toteutus- ja raportointivaiheista. Suunnitteluvaiheessa tunnistetaan kirjallisuuskatsauksen tarve, määritellään tutkimuskysymykset, sekä protokolla katsauksen suorittamiselle. \citep{Kitchenham2007}

Toteutusvaiheessa tehdään aineistohakuja ja valitaan ennaltamääritellyin kriteerein tutkimukselle olennaiset artikkelit. Artikkelien laatua ja sitä kautta luotettavuutta arvioidaan. Tehdyistä hauista kirjataan kaikki kirjallisuuskatsauksen toistamiseen ja sen laadun arviointiin tarvittava tieto, kuten haussa käytetyt hakukoneet, hakusanat, löydettyjen tulosten määrä ja jopa tuloslista. Valitusta aineistosta kerätään tietoa talteen ja sen merkittävyyttä omalle tutkimukselle arvioidaan ennalta määritellyin kriteerein. Kerätyn ja merkittäväksi valitun tiedon perusteella muodostetaan synteesi. Raportointivaiheessa kirjoitetaan tulokset ylös ja arvioidaan tuloksia. \citep{Kitchenham2007}

Tähän kandidaatintyöhön valittiin Kitchenhamin määrittelemä systemaattinen kirjallisuuskatsaus, jotta kerätty aineistolista ja sen pohjalta tehty synteesi olisi tieteellisesti merkittävämpää, sekä mahdollisesti myös muulle tutkimukselle käyttökelpoista.

Kandidaatintyön ohjaajan kanssa sovittiin työn rajaukseksi se, että systemaattinen kirjallisuuskatsaus toteutetaan tässä kandidaatintyössä siten, että haetaan Scopus-tietokannasta vuodesta 1990 lähtien julkaistuja tieteellisiä artikkelejä hakusanoilla "retrospective",  "postmortem analysis", "post-project review" ja "software engineering". Näistä hakusanoista kokeillaan erilaisia yhdistelmiä, ja löytyneiden tulosten määrä, kuin myös listat tuloksista otetaan talteen. Näin tehty SLR on helpompi toistaa jatkossa ja sen oikeellisuus on helpompi todentaa.

Tuloksista ensimmäisen tutkimuskysymyksen kannalta olennaiset, eli sellaiset, jotka kuvaavat jonkinlaista retrospektiiveissä käytettävää menetelmää, kerätään talteen taulukkolaskentaohjelmaan. Artikkelien kaikki lähdeviitteeseen tarvittavat tiedot merkitään ylös. Lisäksi merkitään arvio artikkelin luotettavuudesta ja se, perustuuko artikkelin retrospektiivimenetelmä juurisyyanalyysiin.

\subsection{Käytettävät hakusanat}
Kandidaatintyön kirjoittaja aloitti systemaattisen kirjallisuuskatsauksen kokeilemalla erilaisia yhdistelmiä määritellyistä hakusanoista ja hakemalla niitä eri tavoin artikkeleista. Tavoitteena oli löytää sellainen yhdistelmä, jolla löytyisi rittävä määrä artikkeleita, jotta kirjallisuuskatsauksen otos ei olisi liian suppea ja saattaisi tutkimuksen luotettavuuden sen takia kyseenalaiseksi. Kuitenkin artikkelien määrän tulee olla järkevästi käsiteltävissä kandidaatintyön määrittämän työmäärän puitteissa. Kandidaatintyön tekijä sopi yhdessä työn ohjaajan kanssa sopivan otoskoon olevan noin 100 artikkelia.

Alkuperäinen suunnitelma ol käyttää pelkästään hakusanoja "retrospective",  "postmortem analysis", "post-project review" ja "software engineering". Kuitenkin, koska näillä hakusanoilla olennaisten hakutulosten lukumäärä osoittautui liian pieneksi, otettiin hakusanoiksi mukaan juurisyyanalyysiä kuvaavat hakusanat: "root cause analysis", "rca", "defect cause analysis" ja "dca". Nyt etsimme siis mainintaa juurisyyanalyysin käytöstä retrospektiivin menetelmänä sekä retrospektiiviä että juurisyyanalyysiä käsittelevistä artikkeleista. Siten lähestymme kandidaatintyön aihetta kahdelta suunnalta.

Hakusana "software engineering" osoittautui niin yleinen ja lähinnä aihealuetta kuvaavaksi, etteivät artikkelit yleensä maininneet sitä otsikossaan, abstraktissa tai artikkelin avainsanoissa. Siksi haemme kyseistä hakusanaa kaikista mahdollisista hakukentistä, eikä pelkästään edellämainituista kolmesta. Lisäksi käytämme Scopuksen omaa aihealuerajausta ja haemme ainoastaan artikkeleja tietotekniikan aihealueelta.

Lopulliseksi hakusanaksi muodostui siis:\\
\textit{ALL("software engineering") AND TITLE-ABS-KEY("retrospective" OR "postmortem analysis" OR "pma" OR "post-mortem" OR "post mortem analysis" OR "post-project review" OR "post project review" OR "root cause analysis" OR"rca" OR "defect cause analysis" OR "dca") AND DOCTYPE(ar) AND SUBJAREA(comp) AND PUBYEAR > 1989}\\
Näillä hakusanoilla löytyi yhteensä 108 tulosta.

Kirjallisuuskatsauksessa käytettyjen hakusanojen evoluutio erilaisine kokeiluineen ja niiden toimivuuden arvioinnin kera on annettu kandidaatin työn liitteenä.

\subsection{Tulosten arviointi}
Haun tulokset tallennettiin taulukkolaskentaohjelmaan. Tuloksia arvioitiin eri tavoin käyttämällä artikkelin otsikon, abstraktin ja avainsanojen antamia tietoja.
\clearpage

\subsection{Tulosten rajaus}
Taulukossa 1 on esitetty perusteet, joiden mukaan tuloksia on karsittu pois. Karsinta on tehty otsikon, abstraktin ja artikkelin avainsanojen antamien tietojen perusteella.
\begin{table}
    \begin{tabular}{|p{0.5cm}|p{11.5cm}|p{2cm}|}
        \hline
        \textbf{\#} & \textbf{Karsintaehto} & \textbf{Tulosten määrä} \\ \hline
        0 & Ei rajausta                                                                                                                                               & 108            \\ \hline
        1 & Ei ollut enää Scopus:issa saatavilla arviointihetkellä (Tuloksia käytiin läpi useampana päivänä. Osa artikkeleista ei enää löytynyt myöhemmillä hauilla.) & 106            \\ \hline
        2 & Ei sisältänyt retroa eikä RCA:ta                                                                                                                          & 43             \\ \hline
        3 & Ei sisältänyt retroa ja RCA:n sisältyminen on epävarmaa                                                                                                   & 41             \\ \hline
        4 & Sisältää RCA:n, mutta ei sisällä retroa.                                                                                                                  & 32             \\ \hline
        5 & Esitetty retrospektiivi ei ole yrityksen työntekijöiden (esim kehitystiimin) välinen sessio (vaan esim tutkijoiden jälkeenpäin suorittama)                & 23             \\ \hline
        6 & On epävarmaa, onko esitetty retrospektiivi yrityksen työntekijöiden (esim kehitystiimin) välinen sessio (eikä esim tutkijoiden jälkeenpäin suorittama)    & 20             \\
        \hline
        7 & Artikkelin lukemisen jälkeen kävi ilmi, että jokin yllä mainituista rajauksista oli ymmärretty abstraktin perusteella väärin, eikä artikkeli ollutkaan työn kannalta olennainen. & 10 \\ \hline
    \end{tabular}
    \caption{Systemaattisen kirjallisuuskatsauksen tulosten karsinta}
    \label{tab:karsintaehdot_taulukko}
\end{table}

Arvioitavista 108:sta artikkelista viimeisen karsinnan perusteella luettavaksi jäi 20 artikkelia, jotka olivat kandidaatin työn kannalta potentiaalisesti olennaisia. Artikkeleja lukiessa niiden tietoja päivitettiin. Kymmenen artikkelia putosi näiden päivitysten myötä pois rajausjoukosta, kun artikkelia tarkemmin lukiessa kävi ilmi, ettei se tarjonnutkaan työn kannalta tärkeää tietoa, vaikka abstraktin perusteella näin olisi voinut luulla. Olennaisilta vaikuttavista artikkeleista tehtiin tarkemmat muistiinpanot jatkoanalyysiä varten.

\section{Systemaattinen kirjallisuuskatsauksen tulokset}



\begin{center}
\begin{longtable}{|p{3cm}|p{4cm}|p{4cm}|p{4cm}|}
\caption{Artikkelien sisältämien retrospektiivimenetelmien vaiheet}\\
\hline
  & \textbf{Syötteen kehittäminen} & \textbf{Kausaalianalyysi} & \textbf{Parannusideoiden kehittäminen} \\
\hline
\endfirsthead
\multicolumn{4}{c}%
{\tablename\ \thetable\ -- \textit{Jatkoa edelliseltä sivulta}} \\
\hline
  & \textbf{Syötteen kehittäminen} & \textbf{Kausaalianalyysi} & \textbf{Parannusideoiden kehittäminen} \\
\hline
\endhead
\hline \multicolumn{4}{r}{\textit{Jatkuu seuraavalla sivulla}} \\
\endfoot
\hline
\endlastfoot
	\textbf{\citep{kalinowski2012evidence}} & 1) Ennen tapaamista kerätään sopiva datajoukko 2) Luokitellaan data 3) Pareto-chart:it mainitaan hyvänä tapana datan pääluokkien tunnistamiseen & Syy-seuraus-diagrammin, tarkemmin Fishbonen, piirtäminen mainitaan hyväksi todetuksi tavaksi tunnistaa systemaattisten ongelmien syitä. & - \\ \hline
	\textbf{\citep{Lehtinen2011}} & Fokus-ryhmän tapaaminen & 1) Kerätään syitä anonyymisti sähköpostitse. 2) Muodostetaan kerätyistä syistä suunntattu verkko 3) Brainwriting, brainstorming kokouksessa. Muodostetaan suunnattu verkko. 4) Tunnistetaan juurisyyt säköpostikyselyn avulla & Workshop, jossa parannusideoita kerätään brainwritingia yhdistettynä skeptiseen ja optimistiseen perspektiiviin. \\ \hline
	\textbf{\citep{Bjornson2009} menetelmä 1} & KJ-sessio & Fishbone (ryhmäkeskustelu) & - \\ \hline
	\textbf{\citep{Bjornson2009} menetelmä 2} & KJ-sessio & Kausaalikartan muodostaminen KJ-sessiossa & - \\ \hline
	\textbf{\citep{karlsson2006case}} & 1) Kerätään sopiva datajoukko (valitulle ohjelmistojulkaisulle toteutettuja vaatimuksia) 2) Uudelleenarvioidaan datajoukko 3) Visualisoidaan saatu data & 1) Keskustellaan visualisoidussa datassa ilmenevien ongelmien syistä. Fasilitaattori kirjoittaa keskustelusta muistiinpanoja. 2) Muistiinpanojen pohjalta luodaan taulukko, jolla löydetään eri datapisteille yhteisiä kategorioita (juurisyitä). & Kehitysideoiden kerääminen ja priorisointi \\ \hline
	\textbf{\citep{de2004learning}} & 1) Osallistujat täyttävät etukäteen kyselyn, joka toimii joko muistin virkistyksenä tai retrospektiivin fokuksen löytämisessä 2) KJ-sessio / Strukturoitu haastattelu & Fishbone (ryhmäkeskustelu) & - \\ \hline
	\textbf{\citep{staalhane2004root}} & Tehdään Pareto-analyysi defekteistä & 1) Fishbone (ryhmäkeskustelu) 2) Pisteytyksen avulla tunnistetaan olennaisimmat syyt (juurisyyt) & 1) Valituille syille brainstormataan kehitysideoita, jotka kootaan taulukkoon. Näistä äänestetään olennaisimmat, eli ne, jotka voitaisiin ottaa työn alle. \\ \hline
	\textbf{\citep{staalhane2003post} menetelmä 1} & 1) Fasilitaattori kerää etukäteen projektipäälliköltä taustatietoa projektista 2) KJ-sessio 3) KJ-session tulosten priorisointi & Fishbone & - \\ \hline
	\textbf{\citep{staalhane2003post} menetelmä 2} & 1) Fasilitaattori kerää etukäteen dokumenteista projektin taustatietoja 2) Strukturoitu haastattelu (valkotaulu ja fläppitaulu dokumentointina ja muistin tukena keskustelun ajan) & Syy-seuraus-suhteita tunnistetaan haastattelun aikana (fasilitaattorilla vastuu) & - \\ \hline
	\textbf{\citep{dingsoyr2003extending}} & KJ-sessio & Fishbone &  \\ \hline
	\textbf{\citep{birk2002postmortem}} & 1) Fasilitaattori tutustuu projektiin, käy mm. kaikki dokumentit läpi. 2) Asetetaan retrospektiiville tavoite 3) Kerätään positiivisia ja negatiivisia kokemuksia KJ-sessioilla,  semistrukturoiduilla haastatteluilla tai fasilitoiduilla ryhmäkeskusteluilla. & Fishbone (ryhmäkeskustelu) & - \\ \hline
	\textbf{\citep{card1998learning}} & 1) Valitaan sopiva datajoukko (voidaan tehdä etukäteen) 2) Luokitellaan datajoukko (voidaan tehdä etukäteen) 3) Tunnistetaan olennaisimmat datapisteet Parerto-chart:in avulla & Selvitetään ongelmien juurisyitä keskustelemalla ja tutkitaan löytyykö useammalla ongelmalla samoja syitä. Jos ongelman juurisyy ei löydy triviaalisti, käytetään Fishbone:a (ryhmäkeskusteluineen) apuna. & Fasilitoidun ryhmäkeskustelun avulla löydetään konkreettisia kehitysideoita. Näistä keskitytään niihin, joilla todennäköisimmiin on merkittävä vaikutus ongelmiin. \\ \hline
\end{longtable}
\end{center}




	
	
	

\section{Pohdinta}

\section{Yhteenveto}
