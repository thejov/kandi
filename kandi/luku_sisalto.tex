% --------------------------------------------------------------------

\section{Johdanto}

Ohjelmistoprojekteissa tulee väistämättä vastaan ongelmia. Näiden järjestelmällinen analysointi ja ehkäiseminen jatkuvana osana kehitysprosessia parantaa ohjelmistoprojektin mahdollisuuksia onnistua. Ketterässä ohjelmistokehityksessä kehitystiimi järjestää säännöllisesti iteraation päätteeksi retrospektiivejä. Niissä käydään läpi iteraation aikana löytyneitä ongelmia, sekä pohditaan tapoja ratkaista ne ja siten parantaa tiimin ohjelmistokehitysprosessia. Juurisyyanalyysi tarjoaa rakenteellisen tavan löytää ongelmien aiheuttajia ja voi siten auttaa ehkäisemään näiden ongelmien esiintymistä jatkossa.

Tämä kandidaatintyö käsittelee sitä, miten juurisyyanalyysi soveltuu menetelmäksi ketterän ohjelmistokehitystiimin retrospektiiviin.

Iteraation lopussa pidettävä retrospektiivi on olennainen osa ketterän ohjelmistokehitysprosessin runkoa. Vaikka sen tavoitteet on yleensä määritelty tarkasti kunkin metodologian kuvauksessa, on sen toteutustapa jätetty yleensä tiimin päätettäväksi. Esimerkiksi Scrum-metodologian kuvauksessa on kuvattu retrospektiivien tavoitteet, muttei niiden saavuttamiseen johtavia menetelmiä \citep{ScrumGuide2011}. Tämän kandidaatintyön tarkoituksena selvittää juurysyyanalyysin soveltuvuutta ketterän retrospektiivin menetelmäksi. Olennaista on se, minkälainen juurisyyanalyysi-menetelmä siihen soveltuu. Menetelmän tulee olla erittäin kevyt ja yksinkertainen, jotta sen käyttöönotto lyhyehköissä iteraatio-retrospektiiveissä olisi mielekästä.

Kandidaatintyön tutkimuskysymykset ovat seuraavat:
\begin{enumerate}
\item Minkälaisia menetelmiä aiemmassa kirjallisuudessa on esitetty juurisyyanalyysiä soveltaviin retrospektiiveihin?
\item Kokevatko ketterät ohjelmistokehitystiimit juurisyyanalyysiä soveltavan retrospektiivin tehokkaaksi?
\end{enumerate}

Kandidaatintyön tavoitteena on selvittää järjestelmällisen kirjallisuuskatsauksen \citep{Kitchenham2010} muodossa vastaus ensimmäiseen tutkimuskysymykseen. Toiseen tutkimuskysymykseen haetaan vastausta kenttätestillä.

Kirjallisuustutkimuksen työmäärän pitämiseksi järkevänä, aineistohaut rajoitetaan Scopus-tietokannan tieteellisiin artikkeleihin. Käytettävät hakusanat, joita haetaan artikkelien otsikosta ja avainsanoista, ovat "retrospective", "postmortem analysis", "post-project review" ja "software engineering". Artikkelien tulee olla julkaistu aikaisintaan vuonna 1990. Mikäli näillä rajauksilla löytyy liikaa artikkeleita, tarkennetaan hakua. Liian rajaavien termien, kuten "root cause analysis" tai "agile", käyttöä pyritään kuitenkin välttämään. Näitä käyttämällä saattaisi ohittaa kiinnostavia artikkeleita, jotka käsittelevät olennaisia asioita, mutta eri termejä käyttäen. Eri hakusanayhdistelmillä löytyneiden tulosten määrä kirjataan ylös.

Hakutuloksista olennaisia ovat sellaiset, jotka kuvaavat jonkinlaista menetelmää käytettäväksi retrospektiiveihin. Nämä artikkelit kerätään taulukkolaskentaohjelman viitekokoelmaan, jossa ylläpidetään artikkeleista kaikkia lähdeviitteeseen tarvittavia tietoja, sekä merkinnän siitä, onko artikkelissa kuvattu retrospektiivin menetelmä juurisyyanalyysi. Ne artikkelit, jossa menetelmä sisältää juurisyyanalyysin, päätyvät kandidaatin työhön. Systemaattisen kirjallisuustutkimuksen tarkoituksena on tehdä aineistonhakuprosessista tarkastettava ja toistettava \citep{Kitchenham2007}.

Kandidaatintyössä tehdään synteesi kirjallisuuskatsauksessa kerätyssä aineistossa esitetyistä retrospektiivien juurisyyanalyysi-menetelmistä. On mahdollista, että osa aineistosta käsittelee raskaampaa, suuremman mittakaavan retrospektiiviä, joka pidettäisiin esimerkiksi projektin jälkeieen (post project review). Mikäli synteesin kuvaama menetelmä osoittautuu ketterän ohjelmistokehitystiimin retrospektiiviin liian raskaaksi, karsitaan siitä tähän tarkoitukseen ylimitoitetut kohdat pois. Lopputuloksen tulisi olla sellainen, että tiimi voi suorittaa sen ketterälle retrospektiiville varatussa, verrattain lyhyessä ajassa.

Kandidaatin työn kenttätutkimus suoritetaan ohjelmistoyrityksen ketterän kehitystiimin kanssa. Juurisyyanalyysi-sessiosta sovitaan tiimin kanssa etukäteen. Tiimi valmistelee yhden tai useamman aiheen (ongelman), joita he haluavat käsitellä juurisyyanalyysi-sessiossa. Kandidaatintyön tekijä on ohjaamassa sessiota ja työn valvoja seuraa tapahtumia sivusta. Työkaluna käytetään Aalto-yliopistolla kehitettyä ARCA-tool -web-sovellusta \citep{ArcaTool} mahdollistamaan hajautetun tiimin yhteisen analyysin. Sessio videoidaan ruudunkaappaus-sovelluksella. 

Juurisyyanalyysi-session jälkeen osallistuneilta kerätään palaute ennalta tehdyn palaute-lomakkeen avulla. Lisäksi kerätään suullinen palautte. Kysymykset käsittelevät tiiimin suorittaman juurisyyanalyysin ja siihen käytetyn työkalun helppokäyttöisyyttä ja tehokkuutta. Kenttätutkimuksen analysointi tapahtuu tämän kerätyn aineiston perusteella. Tutkimuksessa käytetty juurisyyanalyysimenetelmä on pelkistetty versio ARCA-menetelmästä \citep{Lehtinen2011}.

Kandidaatintyössä esitellään ensin lyhyesti työn ymmärtämiseen tarvittava teoreettinen tausta, eli määrittellään termit "ketterä retrospektiivi" ja "juurisyyanalyysi". Tämän jälkeen kuvataan työssä käytettyjä menetelmiä, eli järjestelmällistä kirjallisuuskatsausta ja kenttätutkimuksen menetelmiä. Sitten tuodaan julki näillä menetelmillä saadut tulokset ja tehdään näistä tarvittavat johtopäätökset. Kandidaatintyön viimeinen kappale on yhteenveto.

\section{Teoreettinen tausta}
\subsection{Ketterä retrospektiivi}
Ketterissä ohjelmistokehitysmenetelmissä on määritelty, että säännöllisin väliajoin pidetään reflektointi, jossa tiimi pohtii tapoja tulla tehokkaammaksi. Nämä kehitettyjen parannusten perusteella tiimi muuttaa toimintaansa \citep{AgileManifestoPrinciples}. Kevyt retrospektiivisessio on tapa toteuttaa tätä periaatetta. Projektin lopputuloksen kannalta retrospektiivejä on järkevää pitää lyhyin väliajoin. Tällöin tiimin kohtaamat ongelmat ja niihin liittyvät yksityiskohdat ovat tuoreessa muistissa ja parannusehdotukset voidaan ottaa suoraan käyttöön ja siten pyrkiä parantamaan projektin lopputuloksen laatua. \citep{Cockburn2002}

XP- ja Scrum-metodologiassa retrospektiivejä suositellaan pidettäväksi joka iteraation päätteeks. Niissä tiimi reflektoi sitä, missä he ovat viime iteraation aikana onnistuneet hyvin ja missä on vielä kehitettävää. \citep{Lindstrom2004, ScrumGuide2011} Retrospektiivin avulla tiimi saa palautetta työstään. Retrossa voidaan muun muassa pohtia sitä, miten kurinalaisesti tiimi on noudattanut seuraamansa metodologian käytäntöjä ja voisiko niitä räätälöidä sopimaan paremmin tiimin tarpeisiin. \citep{Lindstrom2004}

\subsection{Juurisyyanalyysi}
Juurisyyanalyysi on rakenteellinen tapa tutkia ongelmia ja tunnistaa niiden aiheuttajia. Ideana on se, että korjaamalla syyn aiheuttavia ongelmia voidaan ehkäistä saman ongelman syntymistä uudestaan -- tai ainakin vähentää ongelman uudelleenesiintymisen todennäköisyyttä. \citep{Lehtinen2011} Juurisyyanalyysin avulla voidaan tutkia ongelmien lisäksi myös onnistumisten syitä. \citep{Bjornson2009} Juurisyylle on useita määritelmiä. Se voi tarkoittaa mitä tahansa ongelman aiheuttavaa syytä, syyketjun perimmäisintä syytä tai syytä, johon johtoporras voi vaikuttaa. Juurisyyanalyysin tuloksia voidaan käyttää apuna prosessinkehityksessä. \citep{Lehtinen2011}

\section{Tutkimusmenetelmät}
\subsection{Systemaattinen kirjallisuuskatsaus}
Systemaattisessa kirjallisuuskatsauksessa (SLR, Systematic Literature Review) tehdään kattava arviointi valitusta aiheesta. Arvioinnissa käytetään luotettavaa, tarkkaa ja toistettavisaa olevaa menetelmää. SLR on kirjallisuuskatsauksen muoto, mikä tarkoittaa sitä, että siinä käydään läpi aiempia tutkimuksia, jotka ovat olennaisia omien tutkimuskysymysten valossa. Kerätyn kirjallisuuden pohjalta muodostetaan synteesi.\citep{Kitchenham2007}

SLR:n erityispiirteenä on se, että kirjallisuuden etsiminen, valikointi ja valitun kirjallisuuden analysointi pyritään tekemään toistettavasti ja puolueettomasti. Kitchenham perustelee SLR-menetelmän hyötyä toteamalla, että kirjallisuuskatsaus, joka ei ole SLR:n tapaan perusteellinen ja tasapuolinen ei tarjoa paljoa tieteellistä arvoa. Sen tulokset ovat todennäköisemmin puolueettomia, eli tutkijan kannasta riippumattomia. \citep{Kitchenham2007}

Systemaattinen kirjallisuuskatsaus koostuu kolmesta päävaiheesta: suunnittelu-, toteutus- ja raportointivaiheista. Suunnitteluvaiheessa tunnistetaan kirjallisuuskatsauksen tarve, määritellään tutkimuskysymykset, sekä protokolla katsauksen suorittamiselle. \citep{Kitchenham2007}

Toteutusvaiheessa tehdään aineistohakuja ja valitaan ennaltamääritellyin kriteerein tutkimukselle olennaiset artikkelit. Artikkelien laatua ja sitä kautta luotettavuutta arvioidaan. Tehdyistä hauista kirjataan kaikki kirjallisuuskatsauksen toistamiseen ja sen laadun arviointiin tarvittava tieto, kuten haussa käytetyt hakukoneet, hakusanat, löydettyjen tulosten määrä ja jopa tuloslista. Valitusta aineistosta kerätään tietoa talteen ja sen merkittävyyttä omalle tutkimukselle arvioidaan ennalta määritellyin kriteerein. Kerätyn ja merkittäväksi valitun tiedon perusteella muodostetaan synteesi. Raportointivaiheessa kirjoitetaan tulokset ylös ja arvioidaan tuloksia. \citep{Kitchenham2007}

Tähän kandidaatintyöhön valittiin Kitchenhamin määrittelemä systemaattinen kirjallisuuskatsaus, jotta kerätty aineistolista ja sen pohjalta tehty synteesi olisi tieteellisesti merkittävämpää, sekä mahdollisesti myös muulle tutkimukselle käyttökelpoista.

Systemaattinen kirjallisuuskatsaus toteutetaan tässä kandidaatintyössä siten, että haetaan Scopus-tietokannasta vuodesta 1990 lähtien julkaistuja tieteellisiä artikkelejä hakusanoilla "retrospective",  "postmortem analysis", "post-project review" ja "software engineering". Näistä hakusanoista kokeillaan erilaisia yhdistelmiä, ja löytyneiden tulosten määrä, kuin myös listat tuloksista otetaan talteen. Näin tehty SLR on helpompi toistaa jatkossa ja sen oikeellisuus on helpompi todentaa.

Tuloksista ensimmäisen tutkimuskysymyksen kannalta olennaiset, eli sellaiset, jotka kuvaavat jonkinlaista retrospektiiveissä käytettävää menetelmää, kerätään talteen taulukkolaskentaohjelmaan. Artikkelien kaikki lähdeviitteeseen tarvittavat tiedot merkitään ylös. Lisäksi merkitään arvio artikkelin luotettavuudesta ja se, perustuuko artikkelin retrospektiivimenetelmä juurisyyanalyysiin.


\subsection{Kenttätutkimus}
Kenttätutkimus suoritetaan keskikokoisessa ohjelmistoyrityksessä, jonka ketterälle ohjelmistokehitystiimin retrospektiiviä kandidaatintyön kirjoittaja seuraa. Kyseessä on noin kymmenen jäsenen kokoinen eri maiden välillä toimiva hajautettu tiimi. Retrospektiivissä tiimi pohtii edellisessä sprintissä haasteita aiheuttaneita ongelmia höyhenenkevyttä juurisyyanalyysi-menetelmää käyttäen. Retrospektiivin kommunikaatio tapahtuu videoneuvottelutyökalua käyttäen. Juurisyyanalyysissä käytetään Aalto-yliopistolla kehitettyä ARCA-tool -juurysyyanalyysityökalun \citep{ArcaTool} eri versioita. ARCA-tool on Internetin ylitse toimiva juurisyyanalyysityökalu, joka mahdollistaa kollaboratiivisen ja reaaliaikaisen juurisyyanalyysin tekemisen. Jokainen voi osallistua analyysin tekemiseen omalta työpisteeltään pelkää Internet-selainta käyttäen.

Itse juurisyyanalyysimenetelmän on täytettävä tietyt kriteerit, jotta se soveltuisi tiimin retrospektiiviin. Ensinnäkin tiimi on asettanut retrospektiivin aikaikkunaksi yhden tunnin. Tiimin pitää siis saada valittua juurisyyanalyysimenetelmää käyttämällä konkreettisia tuloksia tunnin pituisen retrospektiivin jälkeen. Tavoitteena on ymmärtää tiimin valitsemat ongelmat ja niiden aiheuttajat syvällisemmin, sekä muodostaa konkreettisia toimia tiimin seuraavan sprintin backlogille näiden ongelmien ehkäisemiseksi jatkossa. Lisäksi menetelmän pitää olla helposti omaksuttava siten, ettei retrospektiiviin osallistuville tarvitse pitää erillistä koulutusta menetelmästä, vaan se voidaan opettaa heille muutamalla lauseella ennen analyysin aloittamista.

Juurisyyanalyysisessiot taltioidaan ruudunkaappaussovelluksella, jolloin videoon sisältyy sekä ARCA-tool:in näkymä koko analyysin ajan, sekä kommunikointiin käytetty videoneuvottelukuva. Taltioinnin tarkoituksena on se, ettei kandidaatintyön kirjoittalta jää huomaamatta yksityiskohtia session kulusta, osallistujien tunnetiloista ja reaktioista tai syyverkon muodostumisesta.

Sessioiden jälkeen kandidaatintyön tekijä kerää osallistujilta palautteen, joka koostuu sekä kyselylomakkeesta, että vapaammasta suullisesta tai pikaviestimen avulla kerätystä palautteesta. Molemmissa esitetyt kysymykset käsittelevät sitä, miten tehokkaaksi ja helppokäyttöiseksi osallistujat kokivat käytetyn juurisyyanalyysimenetelmän ja ARCA-tool:in verrattuna aiempiin prosessinkehitysmenetelmiinsä. Kyselylomake on Google Forms -työkalulla tehty Internet-lomake, jossa on yhteensä 26 kysymystä. 

\subsubsection{Ensimmäinen case}
Ensimmäisen retrospektiivin juurisyyanalyysimenetelmäksi on valittu ARCA-menetelmään perustuva, erittäin yksinkertainen ja pelkistetty,  höyhenenkevyt menetelmä, jonka kuvaan seuraavaksi.

Retrospektiivin lähtöasetelmana on se, että kullakin tiimin jäsenellä on Internet-selaimessaan auki retrospektiiviä varten luotu uusi analyysi ARCA-tool:issa. Tiimi on etukäteen valinnut ongelmat, joihin analyysissä halutaan pureutua.

Fasilitaattorin ohjauksessa tiimin jäsenet keräävät ongelmiin johtaneita syitä. Syiden ja niiden alisyiden kerääminen tehdään kahdessa iteraatioissa. Näissä viiden minuutin pituisissa iteraatioissa tiimin jäsenet lisäävät omatoimisesti ongelmiin ja niiden syihin johtaneita syitä ARCA-tool:issa näkyvään syyverkkoon. Kunkin iteraation jälkeen lisätyt syyt käydään läpi, jotta kaikki saavat yleiskuvan siitä, minkälaisista syistä syyverkko koostuu. Jokainen tiimin jäsen esittelee lisäämänsä syyt. Mikäli läpikäynnin yhteydessä esiintyy uusia syitä, ne lisätään verkkoon. Toisen iteraation jälkeen tiimin scrum master summaa koko syyverkon lyhyesti. 

Viimeinen askel on juurisyiden löytäminen. Fasilitaattori korostaa sitä, että juurisyyt, joita retrospektiivissä haetaan ovat ne syvimmät syyt, joihin tiimi voi vaikuttaa. Mielellään ne ovat sellaisia, joihin vaikuttamisen voi siirtää seuraavan sprintin backlogille. Juurisyiden löytäminen tapahtuu siten, että jokainen tiimin jäsen saa ehdottaa yhtä syytä "tykkäämällä" siitä ARCA-tool:issa. Ne 1-3 syytä, jotka ovat saaneet eniten tykkäyksiä, valitaan juurisyiksi.

Ensimmäisessä case-retrospektiivissä käytetään ARCA-tool:in ensimmäistä versiota, joka kehitettiin 2011-2012 lukuvuonna Aalto-yliopiston Ohjelmistokehitysprojekti-kurssilla (T-76.4115). Siinä on kaikki olennaisimmat juurisyyanalyysin tekemiseen tarvittavat toiminnot, jotka sisältävät muun muassa syyverkon kollaboratiivisen rakentamisen, tiettyjen syiden painottamisen tykkäys-toiminnolla, sekä korjaavien ideoiden lisäämisen.

\subsubsection{Toinen case}
Mikäli toinen RCA-sessio saadaan järjestettyä case-yrityksen tiimin kanssa, käytetään siinä eri juurisyyanalyysimenetelmää ja eri versiota ARCA-tool:ista. Menetelmäksi valitaan systemaattisen kirjallisuuskatsauksen tuloksena syntynyttä synteesiä. Voi olla, että synteesiin täytyy tehdä muutoksia, jotta se sopii tiimin retrospektiiville asettamiin raameihin. Retrospektiivissä käytetään ARCA-toolin uusinta versiota, joka kehitettiin 2012-2013 lukuvuonna Aalto-yliopiston Ohjelmistokehitysprojekti-kurssilla. Siinä on parannettu edellisen version käytettävyyttä, sekä lisätty uutena ominaisuutena mahdollisuus saada tehdystä analyysistä yksinkertaistetun yleiskuvan tarjoava näkymä.

\section{Tulokset}
\subsection{Systemaattinen kirjallisuuskatsaus}
\subsubsection{Käytettävät hakusanat}
Kandidaatintyön kirjoittaja aloitti systemaattisen kirjallisuuskatsauksen kokeilemalla erilaisia yhdistelmiä määritellyistä hakusanoista ja hakemalla niitä eri tavoin artikkeleista. Tavoitteena oli löytää sellainen yhdistelmä, jolla löytyisi rittävä määrä artikkeleita, jotta kirjallisuuskatsauksen otos ei olisi liian suppea ja saattaisi tutkimuksen luotettavuuden sen takia kyseenalaiseksi. Kuitenkin artikkelien määrän tulee olla järkevästi käsiteltävissä kandidaatintyön määrittämän työmäärän puitteissa. Kandidaatintyön tekijä sopi yhdessä työn ohjaajan kanssa sopivan otoskoon olevan noin 100 artikkelia.

Alkuperäinen suunnitelma ol käyttää pelkästään hakusanoja "retrospective",  "postmortem analysis", "post-project review" ja "software engineering". Kuitenkin, koska näillä hakusanoilla olennaisten hakutulosten lukumäärä osoittautui liian pieneksi, otettiin hakusanoiksi mukaan juurisyyanalyysiä kuvaavat hakusanat: "root cause analysis", "rca", "defect cause analysis" ja "dca". Nyt etsimme siis mainintaa juurisyyanalyysin käytöstä retrospektiivin menetelmänä sekä retrospektiiviä että juurisyyanalyysiä käsittelevistä artikkeleista. Siten lähestymme kandidaatintyön aihetta kahdelta suunnalta.

Hakusana "software engineering" osoittautui niin yleinen ja lähinnä aihealuetta kuvaavaksi, etteivät artikkelit yleensä maininneet sitä otsikossaan, abstraktissa tai artikkelin avainsanoissa. Siksi haemme kyseistä hakusanaa kaikista mahdollisista hakukentistä, eikä pelkästään edellämainituista kolmesta. Lisäksi käytämme Scopuksen omaa aihealuerajausta ja haemme ainoastaan artikkeleja tietotekniikan aihealueelta.

Lopulliseksi hakusanaksi muodostui siis:\\
\textit{ALL("software engineering") AND TITLE-ABS-KEY("retrospective" OR "postmortem analysis" OR "pma" OR "post-mortem" OR "post mortem analysis" OR "post-project review" OR "post project review" OR "root cause analysis" OR"rca" OR "defect cause analysis" OR "dca") AND DOCTYPE(ar) AND SUBJAREA(comp) AND PUBYEAR > 1989}\\
Näillä hakusanoilla löytyi yhteensä 108 tulosta.

Kirjallisuuskatsauksessa käytettyjen hakusanojen evoluutio erilaisine kokeiluineen ja niiden toimivuuden arvioinnin kera on annettu kandidaatin työn liitteenä.

\subsubsection{Tulosten arviointi}
Haun tulokset tallennettiin taulukkolaskentaohjelmaan. Tuloksia arvioitiin eri tavoin käyttämällä artikkelin otsikon, abstraktin ja avainsanojen antamia tietoja.
\clearpage

\subsubsection{Tulosten rajaus}
Taulukossa 1 on esitetty perusteet, joiden mukaan tuloksia on karsittu pois. Karsinta on tehty otsikon, abstraktin ja artikkelin avainsanojen antamien tietojen perusteella.
\begin{table}
    \begin{tabular}{|p{12cm}|p{2cm}|}
        \hline
        \textbf{Karsintaehto} & \textbf{Tulosten määrä} \\ \hline
        Ei rajausta                                                                                                                                               & 108            \\ \hline
        Ei ollut enää Scopus:issa saatavilla arviointihetkellä (Tuloksia käytiin läpi useampana päivänä. Osa artikkeleista ei enää löytynyt myöhemmillä hauilla.) & 106            \\ \hline
        Ei sisältänyt retroa eikä RCA:ta                                                                                                                          & 43             \\ \hline
        Ei sisältänyt retroa ja RCA:n sisältyminen on epävarmaa                                                                                                   & 41             \\ \hline
        Sisältää RCA:n, mutta ei sisällä retroa.                                                                                                                  & 32             \\ \hline
        Esitetty retrospektiivi ei ole yrityksen työntekijöiden (esim kehitystiimin) välinen sessio (vaan esim tutkijoiden jälkeenpäin suorittama)                & 23             \\ \hline
        On epävarmaa, onko esitetty retrospektiivi yrityksen työntekijöiden (esim kehitystiimin) välinen sessio (eikä esim tutkijoiden jälkeenpäin suorittama)    & 20             \\
        \hline
    \end{tabular}
    \caption{Systemaattisen kirjallisuuskatsauksen tulosten karsinta}
    \label{tab:karsintaehdot_taulukko}
\end{table}

Arvioitavista 108:sta artikkelista viimeisen karsinnan perusteella luettavaksi jäi 20 artikkelia, jotka olivat kandidaatin työn kannalta potentiaalisesti olennaisia. Artikkeleja lukiessa niiden tietoja päivitettiin. Jotkin artikkelit putosivat näiden päivitysten myötä pois rajausjoukosta, mikäli artikkelia tarkemmin lukiessa kävi ilmi, ettei se tarjonnutkaan työn kannalta tärkeää tietoa, vaikka abstraktin perusteella näin olisi voinut luulla. Olennaisilta vaikuttavista artikkeleista tehtiin tarkemmat muistiinpanot jatkoanalyysiä varten.

\subsection{Kenttätutkimus}
\subsubsection{Ensimmäinen case}
Case-tiimistä Ensimmäiseen retrospektiiviin osallistui yhteensä viisi henkilöä kahdesta eri maasta.
Session jälkeen kerätty kirjallinen, eli Google Forms -kyselylomake koostui seuraavista, alla listatuista kysymyksistä. Tiimille esitetyt kysymykset olivat englanniksi, mutta ne on tässä käännetty suomeksi. Lisäksi niistä on poistettu kaikki viitteet kohdeyrityksen nimeen. Juurisyistä puhuttaessa tarkoitetaan syvimpiä syitä, joihin tiimi voi itse vaikuttaa, eli niitä syitä, jotka valittiin retrospektiivin päätteeksi ehkäistäviksi.

\begin{enumerate}
  \item Mikä on tehtäväsi / työnimikkeesi yrityksessä?
  \item Valitse rooli, joka kuvaa työtehtävääsi parhaiten:
  \begin{itemize}
	\item Johtaja
	\item Tuoteomistaja
	\item Scrum Master
	\item Ohjelmoija
	\item Muu, mikä?
  \end{itemize}
  \item Vastaa asteikolla yhdestä viiteen seuraaviin kysymyksiin ( 1 = erittäin vähäinen / huonoin, 5 = erittäin merkittävä / paras )
   \begin{itemize}
	\item Työ, jonka yrityksenne on käyttänyt valittujen kohdeongelmien tai vastaavien ehkäisemiseen aikaisemmin?
	\item Valitun kohdeongelman sisäinen vaikutus yritykseenne?
	\item Syiden keräämisen helppous?
	\item Juurisyiden tunnistamisen helppous?
	\item Havaittujen syiden oikeellisuus?
	\item Löydettyjen syiden ratkaisemisen vaikutus?
	\item Oma kontribuutioni juurisyyanalyysisessiossa oli?
	\item Juurisyyanalyysisession avoimuus oli?
   \end{itemize}
  \item Vastaa asteikolla yhdestä viiteen seuraaviin kysymyksiin ( 1 = ehdottomasti ei, 5 = ehdottomasti kyllä )
  \begin{itemize}
	\item Verrattuna tiiminne nykyisiin prosessinkehitysmenetelmiin, oliko käytetty RCA-menetelmä mielestäsi kustannustehokas?
	\item Verrattuna tiiminne nykyisiin prosessinkehitysmenetelmiin, oliko käytetyn RCA-menetelmän käyttäminen mielestäsi helppoa?
	\item Verrattuna tiiminne aiempiin tapoihin tunistaa ongelmien takana piileviä syitä, oliko käytetty RCA-menetelmä mielestäsi hyödyllinen?
	\item Verrattuna tiiminne post it -lappujen avulla tehtyyn juurisyyanalyysiin, oliko ARCA-tool:in käyttäminen mielestäsi kustannustehokasta?
	\item Verrattuna tiiminne post it -lappujen avulla tehtyyn juurisyyanalyysiin, oliko ARCA-tool mielestäsi helppokäyttöinen?
	\item Verrattuna tiiminne aiempiin prosessinkehitysmenetelmiin, oliko ARCA-tool mielestäsi hyödyllinen?
	\item Onko ARCA-tool helppokäyttöinen?
	\item Onko ARCA-tool:in käyytö helppo oppia?
	\item Auttoiko ARCA-tool löytämään ongelmien aiheuttajia?
	\item Olisimmeko löytäneet samoja ongelmien aiheuttajia myös ilman ARCA-tool:ia?
	\item Oliko löydettyjen syiden organisointi ARCA-tool:illä vaikeaa?
	\item Helppous löytää syitä, jotka johtavat valittuun kohdeongelmaan oli...
	\item Ratkaistaksenne kohdeongelman, oliko ongelmaan johtavien syiden tunnistaminen hyödyllistä?
	\item Oliko tapa, jota käytettiin kohdeongelman aihettavien syiden tunnistamiseen, hyödyllinen verrattuna yrityksenne aiempiin käytäntöihin?
	\item Oliko helppoa tunnistaa kohdeongelman aiheuttavia syitä?
	\item Oliko ARCA-tool:in käyttäminen prosessinkehityksen kohteiden tunnistamiseen kustannustehokasta verrattuna yrityksenne aiempiin käytäntöihin?
  \end{itemize}
\end{enumerate}

Tämän jälkeen kerätty suullinen palaute kerättiin seuraavaksi listattujen kysymysten avulla. Tiimiläisille esitettyjen kysymysten kieli vaihteli suomen ja englannin välillä riippuen haastateltavan kielitaidosta.
\begin{enumerate}
	\item Verrattuna tiiminne aiempiin prosessinkehitysmenetelmiin (esimerkiksi retrospektiiveissä), oliko käytetty RCA-menetelmä mielestäsi kustannustehokas? Miksi/Miksi ei?
	\item Verrattuna tiiminne aiempiin prosessinkehitysmenetelmiin (esimerkiksi retrospektiiveissä), oliko käytetty RCA-menetelmä mielestäsi mielestä helppo? Miksi/Miksi ei?
	\item Verrattuna siihen, että juurisyyanalyysiä tehdään manuaalisesti, kirjoittamalla käsin paperille (post-iteille) oliko Internetin yli toimiva ARCA-tool mielestäsi kustannustehokas? Miksi/Miksi ei?
	\item Verrattuna siihen, että juurisyyanalyysiä tehdään manuaalisesti, kirjoittamalla käsin paperille (post-iteille) oliko Internetin yli toimiva ARCA-tool mielestäsi helppokäyttöinen? Miksi/Miksi ei?
	\item Tuntuiko ARCA-toolin avulla tehty juurisyyanalyysisessio toimivalta suhteessa yrityksenne aikaisempiin tapoihin tunnistaa ongelman aiheuttajia?
	\item Jos sinulla on vielä kommentteja, parannusideoita, hyviä ja huonoja juttuja pidetystä sessiosta, juurisyyanalyysistä tai ARCA-tool:ista, niin kuulisin niistä mielelläni.
\end{enumerate}

\section{Pohdinta}

\section{Yhteenveto}
