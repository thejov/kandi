% --------------------------------------------------------------------

\section{Johdanto}

Ohjelmistoprojekteissa tulee väistämättä vastaan ongelmia. Näiden järjestelmällinen analysointi ja ehkäiseminen jatkuvana osana kehitysprosessia parantaa ohjelmistoprojektin mahdollisuuksia onnistua. Ketterässä ohjelmistokehityksessä kehitystiimi järjestää säännöllisesti iteraation päätteeksi retrospektiivejä. Niissä käydään läpi iteraation aikana löytyneitä ongelmia, sekä pohditaan tapoja ratkaista ne ja siten parantaa tiimin ohjelmistokehitysprosessia. Juurisyyanalyysi tarjoaa rakenteellisen tavan löytää ongelmien aiheuttajia ja voi siten auttaa ehkäisemään näiden ongelmien esiintymistä jatkossa.

Tämä kandidaatintyö käsittelee sitä, miten juurisyyanalyysi soveltuu menetelmäksi ketterän ohjelmistokehitystiimin retrospektiiviin.

Iteraation lopussa pidettävä retrospektiivi on olennainen osa ketterän ohjelmistokehitysprosessin runkoa. Vaikka sen tavoitteet on yleensä määritelty tarkasti kunkin metodologian kuvauksessa, on sen toteutustapa jätetty yleensä tiimin päätettäväksi. Esimerkiksi Scrum-metodologian kuvauksessa on kuvattu retrospektiivien tavoitteet, muttei niiden saavuttamiseen johtavia menetelmiä \citep{ScrumGuide2011}. Tämän kandidaatintyön tarkoituksena selvittää juurysyyanalyysin soveltuvuutta ketterän retrospektiivin menetelmäksi. Olennaista on se, minkälainen juurisyyanalyysi-menetelmä siihen soveltuu. Menetelmän tulee olla erittäin kevyt ja yksinkertainen, jotta sen käyttöönotto lyhyehköissä iteraatio-retrospektiiveissä olisi mielekästä.

Kandidaatintyön tutkimuskysymykset ovat seuraavat:
\begin{enumerate}
\item Minkälaisia menetelmiä aiemmassa kirjallisuudessa on esitetty juurisyyanalyysiä soveltaviin retrospektiiveihin?
\item Kokevatko ketterät ohjelmistokehitystiimit juurisyyanalyysiä soveltavan retrospektiivin tehokkaaksi?
\end{enumerate}

Kandidaatintyön tavoitteena on selvittää järjestelmällisen kirjallisuuskatsauksen \citep{Kitchenham2010} muodossa vastaus ensimmäiseen tutkimuskysymykseen. Toiseen tutkimuskysymykseen haetaan vastausta kenttätestillä.

Kirjallisuustutkimuksen työmäärän pitämiseksi järkevänä, aineistohaut rajoitetaan Scopus-tiedokannan tieteellisiin artikkeleihin. Käytettävät hakusanat, joita haetaan artikkelien otsikosta ja avainsanoista, ovat "retrospective", "postmortem analysis", "post-project review" ja "software engineering". Artikkelien tulee olla julkaistu aikaisintaan vuonna 1990. Mikäli näillä rajauksilla löytyy liikaa artikkeleita, tarkennetaan hakua. Liian rajaavien termien, kuten "root cause analysis" tai "agile", käyttöä pyritään kuitenkin välttämään. Näitä käyttämällä saattaisi ohittaa kiinnostavia artikkeleita, jotka käsittelevät olennaisia asioita, mutta eri termejä käyttäen. Eri hakusanayhdistelmillä löytyneiden tulosten määrä kirjataan ylös.

Hakutuloksista olennaisia ovat sellaiset, jotka kuvaavat jonkinlaista menetelmää käytettäväksi retrospektiiveihin. Nämä artikkelit kerätään taulukkolaskentaohjelman viitekokoelmaan, jossa ylläpidetään artikkeleista kaikkia lähdeviitteeseen tarvittavia tietoja, sekä merkinnän siitä, onko artikkelissa kuvattu retrospektiivin menetelmä juurisyyanalyysi. Ne artikkelit, jossa menetelmä sisältää juurisyyanalyysin, päätyvät kandidaatin työhön. Systemaattisen kirjallisuustutkimuksen tarkoituksena on tehdä aineistonhakuprosessista tarkastettava ja toistettava.

Kandidaatintyössä tehdään synteesi kirjallisuuskatsauksessa kerätyssä aineistossa esitetyistä retrospektiivien juurisyyanalyysi-menetelmistä. On mahdollista, että osa aineistosta käsittelee raskaampaa, suuremman mittakaavan retrospektiiviä, joka pidettäisiin esimerkiksi projektin jälkeieen (post project review). Mikäli synteesin kuvaama menetelmä osoittautuu ketterän ohjelmistokehitystiimin retrospektiiviin liian raskaaksi, karsitaan siitä tähän tarkoitukseen ylimitoitetut kohdat pois. Lopputuloksen tulisi olla sellainen, että tiimi voi suorittaa sen ketterälle retrospektiiville varatussa, verrattain lyhyessä ajassa.

Kandidaatin työn kenttätutkimus suoritetaan ohjelmistoyrityksen ketterän kehitystiimin kanssa. Juurisyyanalyysi-sessiosta sovitaan tiimin kanssa etukäteen. Tiimi valmistelee yhden tai useamman aiheen (ongelman), joita he haluavat käsitellä juurisyyanalyysi-sessiossa. Kandidaatintyön tekijä on ohjaamassa sessiota ja työn valvoja seuraa tapahtumia sivusta. Työkaluna käytetään Aalto-yliopistolla kehitettyä ARCA-tool -web-sovellusta mahdollistamaan hajautetun tiimin yhteisen analyysin. Sessio videoidaan ruudunkaappaus-sovelluksella. 

Juurisyyanalyysi-session jälkeen osallistuneilta kerätään palaute ennalta tehdyn palaute-lomakkeen avulla. Lisäksi kerätään suullinen palautte. Kysymykset käsittelevät tiiimin suorittaman juurisyyanalyysin ja siihen käytetyn työkalun helppokäyttöisyyttä ja tehokkuutta. Kenttätutkimuksen analysointi tapahtuu tämän kerätyn aineiston perusteella. Tutkimuksessa käytetty juurisyyanalyysimenetelmä on pelkistetty versio ARCA-menetelmästä \citep{Lehtinen2011}.

Kandidaatintyössä esitellään ensin lyhyesti työn ymmärtämiseen tarvittava teoreettinen tausta, eli määrittellään termit "ketterä retrospektiivi" ja "juurisyyanalyysi". Tämän jälkeen kuvataan työssä käytettyjä menetelmiä, eli järjestelmällistä kirjallisuuskatsausta ja kenttätutkimuksen menetelmiä. Sitten tuodaan julki näillä menetelmillä saadut tulokset ja tehdään näistä tarvittavat johtopäätökset. Kandidaatintyön viimeinen kappale on yhteenveto.

\section{Teoreettinen tausta}
\subsection{Ketterä retrospektiivi}


\subsection{Juurisyyanalyysi}

\section{Tutkimusmenetelmät}
\subsection{Systemaattinen kirjallisuuskatsaus}
Systemaattisessa kirjallisuuskatsauksessa (SLR, Systematic Literature Review) tehdään valitusta aiheesta kattava arviointi, jossa käytetään luotettavaa, tarkkaa ja toistettavisaa olevaa menetelmää. SLR on kirjallisuuskatsauksen muoto, eli siinä käydään läpi aiempia tutkimuksia, jotka ovat olennaisia omien tutkimuskysymysten valossa. Kerätyn kirjallisuuden pohjalta muodostetaan synteesi. \citep{Kitchenham2007}

SLR:n erityispiirteenä on se, että kirjallisuuden etsiminen, valikointi ja valitun kirjallisuuden analysointi pyritään tekemään toistettavasti ja puolueettomasti. Kitchenham perustelee SLR-menetelmää toteamalla, että kirjallisuuskatsaus, joka ei ole SLR:n tapaan perusteellinen ja tasapuolinen ei tarjoa paljoa tieteellistä arvoa. Sen tulokset ovat todennäköisemmin puolueetetomia, eli tutkijan kannasta riippumattomia. \citep{Kitchenham2007}

Systemaattinen kirjallisuuskatsaus koostuu kolmesta päävaiheesta: suunnittelu-, toteutus- ja raportointivaiheista. Suunnitteluvaiheessa tunnistetaan kirjallisuuskatsauksen tarve, määritellään tutkimuskysymykset, sekä protokolla katsauksen suorittamiselle. 

Toteutusvaiheessa tehdään aineistohakuja ja valitaan ennaltamääritellyin kriteerein tutkimukselle oleelliset artikkelit. Artikkelien laatua ja sitä kautta luotettavuutta arvioidaan. Tehdyistä hauista kirjataan kaikki kirjallisuuskatsauksen toistamiseen ja sen laadun arviointiin tarvittava tieto, kuten haussa käytetyt hakukoneet, hakusanat, löydettyjen tulosten määrä ja jopa tuloslista. Valitusta aineistosta kerätään tietoa talteen ja sen merkittävyyttä omalle tutkimukselle arvioidaan ennalta määritellyin kriteerein. Kerätyn ja merkittäväksi valitun tiedon perusteella muodostetaan synteesi. Raportointivaiheessa kirjoitetaan tulokset ylös ja arvioidaan tuloksia.

Tähän kandidaatintyöhön valittiin Kitchenhamin määrittelemä systemaattinen kirjallisuuskatsaus, jotta kerätty aineistolista ja sen pohjalta tehty synteesi olisi tieteellisesti merkittävämpää, sekä mahdollisesti myös muulle tutkimukselle käyttökelpoista.

\subsection{Kenttätutkimus}
Kenttätutkimus suoritetaan keskikokoisessa ohjelmistoyrityksessä, jonka ketterälle ohjelmistokehitystiimille kandidaatintyön kirjoittaja pitää retrospektiivin. Retrospektiivissä tiimi pohtii edellisessä sprintissä haasteita aiheuttaneita ongelmia höyhenenkevyttä juurisyyanalyysi-menetelmää käyttäen. Retrospektiivin pitämiseen tuo lisämausteen se, että kyseessä on hajautetusta tiimistä, jossa osa työskentelee eri maassa. Retrospektiivin kommunikaatio tapahtuu videoneuvottelutyökalua käyttäen. Juurisyyanalyysissä käytetään Aalto-yliopistolla kehitettyä ARCA-tool -juurysyyanalyysityökalua. Se toimii Internetin ylitse ja mahdollistaa kollaboratiivisen ja reaaliaikaisen juurisyyanalyysin tekemisen siten, että jokainen voi osallistua analyysin tekemiseen omalta työpisteeltään pelkää Internet-selainta käyttäen.

Itse juurisyyanalyysimenetelmäksi on valittu höyhenenkevyt tiimin kanssa valittuun tunnin aikaikkunaan soveltuva menetelmä. Lähtöasetelmana on se, että kullakin tiimin jäsenellä on Internet-selaimessaan auki retrospektiiviä varten luotu uusi analyysi ARCA-tool:issa. Tiimi on etukäteen kerännyt ne ongelmat, joihin analyysissä halutaan pureutua. Fasilitaattorin ohjauksessa tiimin jäsenet keräävät ongelmiin johtaneita syitä. Syiden ja niiden alisyiden kerääminen tehdään iteraatioissa.

\section{Tulokset}
\subsection{Systemaattinen kirjallisuuskatsaus}
\subsection{Kenttätutkimus}


\section{Pohdinta}

\section{Yhteenveto}
