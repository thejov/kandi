% ---------------------------------------------------------------------
% -------------- PREAMBLE ---------------------------------------------
% ---------------------------------------------------------------------
\documentclass[12pt,a4paper,finnish,oneside]{article}
%\documentclass[12pt,a4paper,finnish,twoside]{article}
%\documentclass[12pt,a4paper,finnish,oneside,draft]{article} % luonnos, nopeampi

% Valitse 'input encoding':
%\usepackage[latin1]{inputenc} % merkistökoodaus, jos ISO-LATIN-1:tä.
\usepackage[utf8]{inputenc}   % merkistökoodaus, jos käytetään UTF8:a
% Valitse 'output/font encoding':
%\usepackage[T1]{fontenc}      % korjaa ääkkösten tavutusta, bittikarttana
\usepackage{ae,aecompl}       % ed. lis. vektorigrafiikkana bittikartan sijasta
% Kieli- ja tavutuspaketit:
\usepackage[english,swedish,finnish]{babel}
% Kurssin omat asetukset aaltosci_t.sty:
\usepackage{aaltosci_t}
% Jos kirjoitat muulla kuin suomen kielellä valitse:
%\usepackage[finnish]{aaltosci_t}           
%\usepackage[swedish]{aaltosci_t}           
%\usepackage[english]{aaltosci_t}           
% Muita paketteja:
\usepackage{alltt}
\usepackage{amsmath}   % matematiikkaa
\usepackage{calc}      % käytetään laskurien (counter) yhteydessä (tiedot.tex)
\usepackage{eurosym}   % eurosymboli: \euro{}
\usepackage{url}       % \url{...}
\usepackage{listings}  % koodilistausten lisääminen
\usepackage{algorithm} % algoritmien lisääminen kelluvina
\usepackage{algorithmic} % algoritmilistaus
\usepackage{hyphenat}  % tavutuksen viilaamiseen liittyvä (hyphenpenalty,...)
\usepackage{supertabular,array}  % useampisivuinen taulukko

% Koko dokumentin kattavia asetuksia:

% Tavutettavia sanoja:
%\hyphenation{vää-rin me-ne-vi-en eri-kois-ten sa-no-jen tavu-raja-ehdo-tuk-set}
% Huomaa, että ylläoleva etsii tarkalleen kyseisiä merkkijonoja, eikä
% ymmärrä taivutuksia. Paikallisesti tekstin seassa voi myös ta\-vut\-taa.

% Rangaistaan tavutusta (ei toimi?! Onko hyphenat-paketti asennettu?)
\hyphenpenalty=10000   % rangaistaan tavutuksesta, 10000=ääretön
\tolerance=1000        % siedetään välejä riveillä
% titlesec-paketti auttaa, jos tämän mukana menee sekaisin

% Tekstiviitteiden ulkoasu.
% Pakettiin natbib.sty/aaltosci.bst liittyen katso esim. 
% http://merkel.zoneo.net/Latex/natbib.php
% jossa selitykset citep, citet, bibpunct, jne.
% Valitse alla olevista tai muokkaa:
\bibpunct{(}{)}{;}{a}{,}{,}    % a = tekijä-vuosi (author-year)
%\bibpunct{[}{]}{;}{n}{,}{,}    % n = numero [1],[2] (numerical style)

% Rivivälin muuttaminen:
\linespread{1.24}\selectfont               % riviväli 1.5
%\linespread{1.24}\selectfont               % riviväli 1, kun kommentoit pois

% ---------------------------------------------------------------------
% -------------- DOCUMENT ---------------------------------------------
% ---------------------------------------------------------------------

\begin{document}

% -------------- Tähän dokumenttiin liittyviä valintoja  --------------

%\raggedright         % Tasattu vain vasemmalta, ei tavutusta
% ----------------- joitakin makroja ----------------------------------
%
% \newcommand{\sinunKomentosi}[argumenttienMäärä]{komennot%
% voiJakaaRiveille%
% jaArgumenttienViittaus#1,#2,#argumenttienMäärä}

% Joskus voi olla tarpeen kommentoida jotakin. Ei suositella. 
% Äläkä unohda lopulliseen! 
\newcommand{\Kommentti}[1]{\fbox{\textbf{OMA KOMMENTTI:} #1}}
% Käyttö: Kilometri on 1024 metriä. \Kommentti{varmista tämä vielä}.
% Eli newcommand:n komentosanan jälkeen hakasaluissa argumenttien lkm,
%  ja argumentteihin viitataa #1, #2, ...

%  Comment out this \DRAFT macro if this version no longer is one!  XXX
%\newcommand{\DRAFT}{\begin{center} {\it DRAFT! \hfill --- \hfill DRAFT!
%\hfill --- \hfill DRAFT! \hfill --- \hfill DRAFT!}\end{center}}

%  Use this \DRAFT macro in the final version - or comment out the 
%  draft-command
% \newcommand{\DRAFT}{~}

% %%%%%%%% MATEMATIIKKA %%%%%%%%%%%%%%%%%

% Määrätty integraali
\newcommand{\myInt}[4]{%
\int_{#1}^{#2} #3 \, \textrm{d}{#4}}

% http://kapsi.fi/jks/satfaq/
%\newcommand{\vii}{\mathop{\Big/}}
%\newcommand{\viiva}[2]{\vii\limits_{\!\!\!\!{#1}}^{\>\,{#2}}}
%%\[ \intop_0^{10} \frac{x}{x^2+1} \,\mathrm{d}x
%%= \viiva{0}{10} \frac{1}{2}\ln(x^2+1) \]

% matht.sty, Simo K. Kivelä, 01.01.2002, 07.04.2004, 19.11.2004, 21.02.2005
% Kokoelma matemaattisten lausekkeiden kirjoittamista helpottavia
% määrittelyjä.

% 07.04.2004 Muutama lisäys ja muutos tehty: \ii, \ee, \dd, \der,
% \norm, \abs, \tr.
%
% 19.11.2004 Korjattu määrittelyjä: \re, \im, \norm;
% lisätty \trp (transponointi), \hrm (hermitointi), \itgr (rakenteellinen
% integraali), ympäristö Cmatrix (hakasulkumatriisi);
% vanha transponointi \tr on mukana edelleen, mutta ei suositella.

% Pakotettu rivinvaihto, joka voidaan tarvittaessa määritellä
% uudelleen: 

%\newcommand{\nl}{\newline}

% Logiikan symboleja (<=> ja =>) hieman muunnettuina:

%\newcommand{\ifftmp}{\;\Leftrightarrow\;}
%\newcommand{\impltmp}{\DOTSB\;\Rightarrow\;}

% 'siten, että' -lyhenne ja hattupääyhtäläisyysmerkki vastaavuuden
% osoittamiseen: 

%\newcommand{\se}{\quad \text{siten, että} \quad}
%\newcommand{\vs}{\ {\widehat =}\ }

% Lukujoukkosymbolit:

%\newcommand{\N}{\ensuremath{\mathbb N}}
%\newcommand{\Z}{\ensuremath{\mathbb Z}}
%\newcommand{\Q}{\ensuremath{\mathbb Q}}
%\newcommand{\R}{\ensuremath{\mathbb R}}
%\newcommand{\C}{\ensuremath{\mathbb C}}

% Reaali- ja imaginaariosa, imaginaariyksikkö:

%\newcommand{\re}{\operatorname{Re}}
%\newcommand{\im}{\operatorname{Im}}
%\newcommand{\ii}{\mathrm{i}}

% Differentiaalin d, Neperin luku:

%\newcommand{\dd}{\mathrm{d}}
%\newcommand{\ee}{\mathrm{e}}

% Vektorimerkintä, joka voidaan tarvittaessa määritellä uudelleen
% (tämä tekee vektorit lihavoituina):

%\newcommand{\V}[1]{{\mathbf #1}}

% Kulmasymboli:

%\renewcommand{\angle}{\sphericalangle}

% Vektorimerkintä, jossa päälle pannaan iso nuoli;
% esimerkiksi \overrightarrow{AB} (tämmöisiä olemassaolevien
% symbolien uudelleenmäärittelyjä ei kyllä pitäisi tehdä):

%\renewcommand{\vec}[1]{\overrightarrow{#1}}

% Vektoreiden vastakkaissuuntaisuus:

%\newcommand{\updownarrows}{\uparrow\negthinspace\downarrow}

% Itseisarvot ja normi:

%\newcommand{\abs}[1]{{\left\vert#1\right\vert}}
%\newcommand{\norm}[1]{{\left\Vert #1 \right\Vert}}

% Transponointi ja hermitointi:

%\newcommand{\trp}[1]{{#1}\sp{\operatorname{T}}}
%\newcommand{\hrm}[1]{{#1}\sp{\operatorname{H}}}

% Vanha transponointi; jäljellä yhteensopivuussyistä, ei syytä käyttää.
%\newcommand{\tr}{{}^{\text T}}

% Arcus- ja area-funktiot, jossa päähaara osoitetaan nimen päälle
% vedetyllä vaakasuoralla viivalla (alkaa olla vanhentunutta,
% voitaisiin luopua):

%\newcommand{\arccot}{\operatorname{arccot}}
%\newcommand{\asin}{\operatorname{\overline{arc}sin}}
%\newcommand{\acos}{\operatorname{\overline{arc}cos}}
%\newcommand{\atan}{\operatorname{\overline{arc}tan}}
%\newcommand{\acot}{\operatorname{\overline{arc}cot}}

%\newcommand{\arsinh}{\operatorname{arsinh}}
%\newcommand{\arcosh}{\operatorname{arcosh}}
%\newcommand{\artanh}{\operatorname{artanh}}
%\newcommand{\arcoth}{\operatorname{arcoth}}
%\newcommand{\acosh}{\operatorname{\overline{ar}cosh}}

% Signum, syt, pyj:

%\newcommand{\sg}{\operatorname{sgn}}
%\renewcommand{\gcd}{\operatorname{syt}}
%\newcommand{\lcm}{\operatorname{pyj}}

% Lyhennemerkintöjä: derivaatta, osittaisderivaatta, gradientti,
% derivaattaoperaattori, vektorin komponentti, integraalin ylä- ja
% alasumma, Suomessa (ja Saksassa?) käytetty integraalin sijoitus-
% merkintä, integraali (rakenteellinen määrittely):

%\newcommand{\der}[2]{\frac{\dd #1}{\dd #2}}
%\newcommand{\osder}[2]{\frac{\partial #1}{\partial #2}}
%\newcommand{\grad}{\operatorname{grad}}
%\newcommand{\Df}{\operatorname{D}} 
%\newcommand{\comp}{\operatorname{comp}}
%\newcommand{\ys}[1]{\overline S_{#1}}
%\newcommand{\as}[1]{\underline S_{#1}}
%\newcommand{\sijoitus}[2]{\biggl/_{\null\hskip-6pt #1}^{\null\hskip2pt #2}} 
%\newcommand{\itgr}[4]{\int_{#1}^{#2}#3\,\dd #4}

% Matriiseja, joille voidaan antaa alkioiden sijoittamista sarakkeen
% vasempaan tai oikeaan reunaan tai keskelle osoittava lisäparametri
% (l, r tai c); ympärillä kaarisulut, hakasulut, pystyviivat (determinantti)
% tai ei mitään;
% esimerkiksi \begin{cmatrix}{ll}1 & -1 \\ -1 & 1 \end{cmatrix}:

%\newenvironment{cmatrix}[1]{\left(\begin{array}{#1}}{\end{array}\right)}
%\newenvironment{Cmatrix}[1]{\left[\begin{array}{#1}}{\end{array}\right]}
%\newenvironment{dmatrix}[1]{\left|\begin{array}{#1}}{\end{array}\right|}
%\newenvironment{ematrix}[1]{\begin{array}{#1}}{\end{array}}

% Kaunokirjoitussymboli:

%\newcommand{\Cal}{\mathcal}

% Isokokoinen summa:

%\newcommand{\dsum}[2]{{\displaystyle \sum_{#1}^{#2}}}

% Tuplaintegraali umpinaisen pinnan yli; korvataan jos parempi löytyy:
%\newcommand\oiint{\begingroup
% \displaystyle \unitlength 1pt
% \int\mkern-7.2mu
% \begin{picture}(0,3)
%   \put(0,3){\oval(10,8)}
% \end{picture}
% \mkern-7mu\int\endgroup}
       % Haetaan joitakin makroja

% Kieli:
% Kielesi, jolla kandidaatintyön kirjoitat: finnish, swedish, english.
% Tästä tulee mm. tietyt otsikkonimet ja kuva- ja taulukkoteksteihin 
% (Kuva, Figur, Figure), (Taulukko, Tabell, Table) sekä oikea tavutus.
\selectlanguage{finnish}
%\selectlanguage{swedish}
%\selectlanguage{english}

% Sivunumeroinnin kanssa pieniä ristiriitaisuuksia.
% Toimitaan pääosin lähteen "Kirjoitusopas" luvun 5.2.2 mukaisesti.
% Sivut numeroidaan juoksevasti arabialaisin siten että 
% ensimmäiseltä nimiölehdeltä puuttuu numerointi.
\pagestyle{plain}
\pagenumbering{arabic}
% Muita tapoja: kandiohjeet: ei numerointia lainkaan ennen tekstiosaa
%\pagestyle{empty}
% Muita tapoja: kandiohjeet: roomalainen numerointi alussa ennen tekstiosaa
%\pagestyle{plain}
%\pagenumbering{roman}        % i,ii,iii, samalla alustaa laskurin ykköseksi

% ---------------------------------------------------------------------
% -------------- Luettelosivut alkavat --------------------------------
% ---------------------------------------------------------------------

% -------------- Nimiölehti ja sen tiedot -----------------------------
%
% Nimiölehti ja tiivistelmä kirjoitetaan seminaarin mukaan joko
% suomeksi tai ruotsiksi (ellei erityisesti kielenä ole englanti). 
% Tiivistelmän voi suomen/ruotsin lisäksi kirjoittaa halutessaan
% myös englanniksi. Eli tiivistelmiä tulee yksi tai kaksi kpl.
%
% "\MUUTTUJA"-kohdat luetaan aaltosci_t.sty:ä varten.

\author{Juha Viljanen}

% Otsikko nimiölehdelle. Yleensä sama kuin seuraavana oleva \TITLE, 
% mutta jos nimiölehdellä tarvetta "kaksiosaiselle" kaksiriviselle
\title{Juurisyyanalyysin soveltuvuus ohjelmistoprojektien retrospektiiveissä}
% 2-osainen otsikko:
%\title{\LaTeX{}-pohja kandidaatintyölle \\[5mm] Pitkiä rivejä kokeilun vuoksi.}

% Otsikko tiivistelmään. Jos lisäksi engl. tiivistelmä, niin viimeisin:
\TITLE{\LaTeX{}-pohja kandidaatintyötä varten ohjeiden kera ja varuilta %
kokeillaan vähän ylipitkää otsikkoa}
%\TITLE{\LaTeX{} för kandidatseminariet med jättelång rubrik som fortsätter och
% fortsätter ännu}
\ENTITLE{\LaTeX{} template for Bachelor thesis with a pretty long title %
line which continues ynd continues}
% 2-osainen otsikko korvataan täällä esim. pisteellä:
%\TITLE{\LaTeX{}-pohja kandidaatintyölle. Pitkiä rivejä kokeilun vuoksi.}

% Ohjaajan laitos suomi/ruotsi ja tarvittaessa eng (tiivistelmän kieli/kielet)
\DEPT{SoberIT}
% suomi:
%\DEPT{Tietotekniikan laitos}               % T
%\DEPT{Tietojenkäsittelytieteen laitos}     % TKT
%\DEPT{Mediatekniikan laitos}               % ME
% ruotsi:
%\DEPT{Institutionen för datateknik}        % T
%\DEPT{Institutionen för datavetenskap}     % TKT
%\DEPT{Institutionen för mediateknik}       % ME
% englanti:
%\ENDEPT{Department of Computer Science Engineering}     % T
%\ENDEPT{Department of Information and Computer Science} % TKT
%\ENDEPT{Department of Media Technology}                 % ME

% Vuosi ja päivämäärä, jolloin työ on jätetty tarkistettavaksi.
\YEAR{2013}
\DATE{21. tammikuuta 2013}
%\DATE{31. helmikuuta 2011}
%\DATE{Den 31 februari 2011}
\ENDATE{February 31, 2013}

% Kurssin vastuuopettaja ja työsi ohjaaja(t)
\SUPERVISOR{Ma professori Tomi Janhunen}
\INSTRUCTOR{DI Timo Lehtinen}
%\INSTRUCTOR{Ohjaajantitteli Timo Lehtinen, ToinenTitt Matti Meikäläinen}
% DI       // på svenska DI diplomingenjör
% TkL      // TkL teknologie licentiat
% TkT      // TkD teknologie doctor
% Dosentti Dos. // Doc. Docent
% Professori Prof. // Prof. Professor
% 
% Jos tiivistelmä englanniksi, niin:
\ENSUPERVISOR{Professor (pro tem) Tomi Janhunen}
\ENINSTRUCTOR{Your instructor, titleOfInstructor}
% M.Sc. (Tech)  // M.Sc. (Eng)
% Lic.Sc. (Tech)
% D.Sc. (Tech)   // FT filosofian tohtori, PhD Doctor of Philosophy
% Docent
% Professor

% Kirjoita tänne HOPS:ssa vahvistettu pääaineesi.
% Pääainekoodit TIK-opinto-oppaasta.

\PAAAINE{Ohjelmistotuotanto ja -liiketoiminta}
\CODE{T3003}
%
%\PAAAINE{Tietoliikenneohjelmistot}
%\CODE{T3005}
%
%\PAAAINE{WWW-teknologiat} % vuodesta 2010
%\CODE{IL3012}
%
%\PAAAINE{Mediatekniikka} % vuoteen 2010, kts. seur.
%\CODE{T3004}
%
%\PAAAINE{Mediatekniikka} % vuodesta 2010, kts. edell.
%\CODE{IL3011}
%
%\PAAAINE{Tietojenkäsittelytiede} % vuodesta 2010
%\CODE{IL3010}
%
%\PAAAINE{Informaatiotekniikka} % vuoteen 2010
%\CODE{T3006}
%
%\PAAAINE{Tietojenkäsittelyteoria} % vuoteen 2010
%\CODE{T3002}
%
%\PAAAINE{Ohjelmistotekniikka}
%\CODE{T3001}

% Avainsanat tiivistelmään. Tarvittaessa myös englanniksi:

\KEYWORDS{juurisyyanalyysi, retrospektiivi}
\ENKEYWORDS{key, words, the same as in FIN/SWE}

% Tiivistelmään tulee opinnäytteen sivumäärä.
% Kirjoita lopulliset sivumäärät käsin tai kokeile koodia. 
%
% Ohje 29.8.2011 kirjaston henkilökunnalta:
%   - yhteissivumäärä nimiölehdeltä ihan loppuun
%   - "kaikkien yksinkertaisin ja yksiselitteisin tapa"
%
% VANHA // Ohje 14.11.2006, luku 4.2.5:
% VANHA // - sivumäärä = tekstiosan (alkaen johdantoluvusta) ja 
% VANHA //  lähdeluettelon sivumäärä, esim. "20"
% VANHA // - jos liitteet, niin edellisen lisäksi liitteiden sivumäärä,
% VANHA //  tyyli "20 + 5", jossa 5 sivua liitteitä 
% VANHA // - HUOM! Tässä oletuksena sivunumerointi alkaa nimiölehdestä 
% VANHA //  sivunumerolla 1. %   Toisin sanoen, viimeisen lähdeluettelosivun 
% VANHA //  sivunumero EI ole sivujen määrä vaan se pitää laskea tähän käsin

\PAGES{Kirjoita tähän oikea määrä, tässä esimerkissä 23}
%\PAGES{23}  % kaikki sivut laskettuna nimiölehdestä lähdeluettelon tai 
             % mahdollisten liitteiden loppuun. Tässä 23 sivua

%\thispagestyle{empty}  % nimiölehdellä ei ole sivunumerointia; tyylin mukaan ei tehdäkään?!

\maketitle             % tehdään nimiölehti

% -------------- Tiivistelmä / abstract -------------------------------
% Lisää abstrakti kandikielellä (ja halutessasi lisäksi englanniksi).

% Edelleen sivunumerointiin. Eräs ohje käskee aloittaa sivunumeroiden
% laskemisen nimiösivulta kuitenkin niin, että sille ei numeroa merkitä
% (Kauranen, luku 5.2.2). Näin ollen ensimmäisen tiivistelmän sivunumero
% on 2. \maketitle komento jotenkin kadottaa sivunumeronsa.
\setcounter{page}{2}    % sivunumeroksi tulee 2

% Tiivistelmät tehdään viimeiseksi. 
%
% Avainsanojen lista pitää merkitä main.tex-tiedoston kohtaan \KEYWORDS.

\begin{fiabstract}

Ketterä tuotekehityksessä ohjelmistokehitystiimi pitää säännöllisin väliajoin retrospektiivin, jonka avulla se pyrkii parantamaan omaa työskentelyprosessiaan. Ketterää ohjelmistokehitystä kuvaavat metodologiat eivät kuitenkaan määrittele retrospektiivin toteutustapaa. Juurisyyanalyysi tarjoaa rakenteellisen tavan tutkia ongelmien aiheuttajia ja voi siten sopia hyvin retrospektiivin ongelmanratkaisuun. 

Tässä kandidaatintyössä pyritään löytämään systemaattisen kirjallisuuskatsauksen avulla vastaus siihen, minkälaisia menetelmiä aiemmassa kirjallisuudessa on esitetty juurisyyanalyysiä soveltaviin retrospektiiveihin. Menetelmät kuvataan, niitä vertaillaan keskenään ja niitä analysoidaan. Menetelmien perusteella kehitetään synteesi, joka kuvaa ketterään retrospektiiviin soveltuvan menetelmän.

Iso osa artikkeleista käytti olennaisten ongelmien ja onnistumisten tunnistamiseen KJ-menetelmää, kausaalianalyysiin fasilitoitua ryhmäkeskustelua käyttäen Ishikawan kalanruotodiagrammia ongelmien syy-seuraus-suhteiden esittämiseen. Uusimmat artikkelit käyttivät kausaalianalyysiin KJ-menetelmän tapaista osallistavaa menetelmää ja suunnattua verkkoa ongelmien syy-seuraus-esitystapana. Alle puolet kuvatuista menetelmistä sisälsi kehitysehdotuksien kehittämisvaiheen.

Muodostettu synteesi on seuraava. Kausaalianalyysin syöte kehitetään käyttäen KJ-menetelmää ja olennaisten asioiden valintaan käsiäänestystä. Kausaalianalyysissä muodostetaan KJ-menetelmää käyttäen suunnattu verkko visualisoimaan syy-seuraus-suhteet. Juurisyyt valitaan käsiäänestyksellä. Brainstoring-menetelmän avulla muodostetaan parannusideataulukko, josta toteutettavat ideat äänestetään käsiäänestyksellä.

\end{fiabstract}

\newpage                       % pakota sivunvaihto

% -------------- Sisällysluettelo / TOC -------------------------------

\tableofcontents

\label{pages:prelude}
\clearpage                     % kappale loppuu, loput kelluvat tänne, sivunv.
%\newpage

% -------------- Symboli- ja lyhenneluettelo -------------------------
% Lyhenteet, termit ja symbolit.
% Suositus: Käytä vasta kun paljon symboleja tai lyhenteitä.
%
% -------------- Symbolit ja lyhenteet --------------
%
% Suomen kielen lehtorin suositus: vasta kun noin 10-20 symbolia
% tai lyhennettä, niin käytä vasta sitten.
%
% Tämä voi puuttuakin. Toisaalta jos käytät paljon akronyymejä,
% niin ne kannattaa esitellä ensimmäisen kerran niitä käytettäessä.
% Muissa tapauksissa lukija voi helposti tarkistaa sen tästä
% luettelosta. Esim. "Automaattinen tietojenkäsittely (ATK) mahdollistaa..."
% "... ATK on ..."

\addcontentsline{toc}{section}{Käytetyt symbolit ja lyhenteet}

\section*{Käytetyt symbolit ja lyhenteet}
%?? Käytetyt lyhenteet ja termit ??
%?? Käytetyt lyhenteet / termit / symbolit ??
%\section*{Abbreviations and Acronyms}

\begin{center}
\begin{tabular}{p{0.2\textwidth}p{0.65\textwidth}}
3GPP  & 3rd Generation Partnership Project; Kolmannen sukupolven 
matkapuhelupalvelu \\ 
ESP & Encapsulating Security Payload; Yksi IPsec-tietoturvaprotokolla \\ 
$\Omega_i$ & hilavitkuttimen kulmataajuus \\
$\mathbf{m}_{ic}$ & hilavitkutinjärjestelmän $i$ painokertoimet \\
\end{tabular}
\end{center}

\vspace{10mm}

Tähän voidaan listata kaikki työssä käytetyt lyhenteet. Lyhenteistä
annetaan selityksenä sekä alkukielinen termi kokonaisuudessaan
(esim. englanninkielinen lyhenne avattuna sanoiksi) että sama
suomeksi. Jos suoraa käännöstä ei ole tai sellaisesta on vaikea saada
sujuvaa, voi käännöksen sijaan antaa selityksen siitä, mitä kyseinen
käsite tarkoittaa. Jos lyhenteitä ei esiinny työssä paljon, ei tätä
osiota tarvita ollenkaan. Yleensä luettelo tehdään, kun lyhenteitä on
10--20 tai enemmän. Vaikka lyhenteet annettaisiinkin tässä
keskitetysti, ne pitää silti avata sekä suomeksi että alkukielellä
myös itse tekstissä, kun ne esiintyvät siellä ensi kertaa.  Käytetyt
lyhenteet -osion voi nimetä myös ``Käytetyt lyhenteet ja termit'', jos
luettelossa on sekä lyhenteitä että muuta käsitteenmäärittelyä.

\textbf{TIK.kand suositus: Lisää lyhenne- tai symbolisivu, kun se
  näyttää luontevalta ja järkevältä. (Käytä vasta kun lyhenteitä yli 10.)}

%Jos tarvitset useampisivuista taulukkoa, kannattanee käyttää 
%esim. \verb!supertabular*!-ympäristöä, josta on kommentoitu esimerkki
%toisaalla tekstiä.


 
%\clearpage                     % luku loppuu, loput kelluvat tänne
\newpage

% -------------- Kuvat ja taulukot ------------------------------------
% Kirjoissa (väitöskirja) on usein tässä kuvien ja taulukoiden listaus.
% Suositus: Ei kandityöhön.

% -------------- Alkusanat --------------------------------------------
% Suositus: ÄLÄ käytä kandidaatintyössä. Jos käytät, niin omalle 
% sivulleen käyttäen tarvittaessa \newpage
%
%% --------------- Alkusanat -------------------------------------------
%
% Suositus: Älä käytä kandidaatintyössä.
%

\addcontentsline{toc}{section}{Alkusanat}

\section*{Alkusanat}
%\section*{Förord}
%\section*{Acknowledgements}

Alkusanoissa voi kiittää tahoja, jotka ovat merkittävästi edistäneet
työn valmistumista. Tällaisia voivat olla esimerkiksi yritys, jonka
tietokantoja, kontakteja tai välineistöä olet saanut käyttöösi,
haastatellut henkilöt, ohjaajasi tai muut opettajat ja myös
henkilökohtaiset kontaktisi, joiden tuki on ollut korvaamatonta työn
kirjoitusvaiheessa. Alkusanat jätetään tyypillisesti pois
kandidaatintyöstä, joka on laajuudeltaan vielä niin suppea, ettei
kiiteltäviä tahoja luontevasti ole.

\textbf{TIK.kand suositus: Älä käytä tällaista lukua.}

\vskip 10mm
Espoossa 31. helmikuuta 2011
\vskip 15mm
Teemu Teekkari


%\clearpage                     % luku loppuu, loput kelluvat tänne
%\newpage                       % pakota sivunvaihto
%
%SH: Alkusanoissa voi kiittää tahoja, jotka ovat merkittävästi edistäneet
% työn valmistumista. Tällaisia voivat olla esimerkiksi yritys, jonka
% tietokantoja, kontakteja tai välineistöä olet saanut käyttöösi,
% haastatellut henkilöt, ohjaajasi tai muut opettajat ja myös
% henkilökohtaiset kontaktisi, joiden tuki on ollut korvaamatonta työn
% kirjoitusvaiheessa. Alkusanat jätetään tyypillisesti pois
% kandidaatintyöstä, joka on laajuudeltaan vielä niin suppea, ettei
% kiiteltäviä tahoja luontevasti ole.

% ---------------------------------------------------------------------
% -------------- Tekstiosa alkaa --------------------------------------
% ---------------------------------------------------------------------

% Muutetaan tarvittaessa ala- ja ylätunnisteet
%\pagestyle{headings}          % headeriin lisätietoja
%\pagestyle{fancyheadings}     % headeriin lisätietoja
%\pagestyle{plain}             % ei header, footer: sivunumero

% Sivunumerointi, jos käytetty 'roman' aiemmin
% \pagenumbering{arabic}        % 1,2,3, samalla alustaa laskurin ykköseksi
% \thispagestyle{empty}         % pyydetty ensimmäinen tekstisivu tyhjäksi

% input-komento upottaa tiedoston 
% --------------------------------------------------------------------

\section{Johdanto}

Ohjelmistoprojekteissa tulee väistämättä vastaan ongelmia. Näiden järjestelmällinen analysointi ja ehkäiseminen jatkuvana osana kehitysprosessia parantaa ohjelmistoprojektin mahdollisuuksia onnistua. Ketterässä ohjelmistokehityksessä kehitystiimi järjestää säännöllisesti iteraation päätteeksi retrospektiivejä. Niissä käydään läpi iteraation aikana löytyneitä ongelmia, sekä pohditaan tapoja ratkaista ne ja siten parantaa tiimin ohjelmistokehitysprosessia. Juurisyyanalyysi tarjoaa rakenteellisen tavan löytää ongelmien aiheuttajia ja voi siten auttaa ehkäisemään näiden ongelmien esiintymistä jatkossa.

Tämä kandidaatintyö käsittelee sitä, miten juurisyyanalyysi soveltuu menetelmäksi ketterän ohjelmistokehitystiimin retrospektiiviin.

Iteraation lopussa pidettävä retrospektiivi on olennainen osa ketterän ohjelmistokehitysprosessin runkoa. Vaikka sen tavoitteet on yleensä määritelty tarkasti kunkin metodologian kuvauksessa, on sen toteutustapa jätetty yleensä tiimin päätettäväksi. Esimerkiksi Scrum-metodologian kuvauksessa on kuvattu retrospektiivien tavoitteet, muttei niiden saavuttamiseen johtavia menetelmiä \citep{ScrumGuide2011}. Tämän kandidaatintyön tarkoituksena selvittää juurysyyanalyysin soveltuvuutta ketterän retrospektiivin menetelmäksi. Olennaista on se, minkälainen juurisyyanalyysi-menetelmä siihen soveltuu. Menetelmän tulee olla erittäin kevyt ja yksinkertainen, jotta sen käyttöönotto lyhyehköissä iteraatio-retrospektiiveissä olisi mielekästä.

Kandidaatintyön tutkimuskysymykset ovat seuraavat:
\begin{enumerate}
\item Minkälaisia menetelmiä aiemmassa kirjallisuudessa on esitetty juurisyyanalyysiä soveltaviin retrospektiiveihin?
\item Kokevatko ketterät ohjelmistokehitystiimit juurisyyanalyysiä soveltavan retrospektiivin tehokkaaksi?
\end{enumerate}

Kandidaatintyön tavoitteena on selvittää järjestelmällisen kirjallisuuskatsauksen \citep{Kitchenham2010} muodossa vastaus ensimmäiseen tutkimuskysymykseen. Toiseen tutkimuskysymykseen haetaan vastausta kenttätestillä.

Kirjallisuustutkimuksen työmäärän pitämiseksi järkevänä, aineistohaut rajoitetaan Scopus-tiedokannan tieteellisiin artikkeleihin. Käytettävät hakusanat, joita haetaan artikkelien otsikosta ja avainsanoista, ovat "retrospective", "postmortem analysis", "post-project review" ja "software engineering". Artikkelien tulee olla julkaistu aikaisintaan vuonna 1990. Mikäli näillä rajauksilla löytyy liikaa artikkeleita, tarkennetaan hakua. Liian rajaavien termien, kuten "root cause analysis" tai "agile", käyttöä pyritään kuitenkin välttämään. Näitä käyttämällä saattaisi ohittaa kiinnostavia artikkeleita, jotka käsittelevät olennaisia asioita, mutta eri termejä käyttäen. Eri hakusanayhdistelmillä löytyneiden tulosten määrä kirjataan ylös.

Hakutuloksista olennaisia ovat sellaiset, jotka kuvaavat jonkinlaista menetelmää käytettäväksi retrospektiiveihin. Nämä artikkelit kerätään taulukkolaskentaohjelman viitekokoelmaan, jossa ylläpidetään artikkeleista kaikkia lähdeviitteeseen tarvittavia tietoja, sekä merkinnän siitä, onko artikkelissa kuvattu retrospektiivin menetelmä juurisyyanalyysi. Ne artikkelit, jossa menetelmä sisältää juurisyyanalyysin, päätyvät kandidaatin työhön. Systemaattisen kirjallisuustutkimuksen tarkoituksena on tehdä aineistonhakuprosessista tarkastettava ja toistettava.

Kandidaatintyössä tehdään synteesi kirjallisuuskatsauksessa kerätyssä aineistossa esitetyistä retrospektiivien juurisyyanalyysi-menetelmistä. On mahdollista, että osa aineistosta käsittelee raskaampaa, suuremman mittakaavan retrospektiiviä, joka pidettäisiin esimerkiksi projektin jälkeieen (post project review). Mikäli synteesin kuvaama menetelmä osoittautuu ketterän ohjelmistokehitystiimin retrospektiiviin liian raskaaksi, karsitaan siitä tähän tarkoitukseen ylimitoitetut kohdat pois. Lopputuloksen tulisi olla sellainen, että tiimi voi suorittaa sen ketterälle retrospektiiville varatussa, verrattain lyhyessä ajassa.

Kandidaatin työn kenttätutkimus suoritetaan ohjelmistoyrityksen ketterän kehitystiimin kanssa. Juurisyyanalyysi-sessiosta sovitaan tiimin kanssa etukäteen. Tiimi valmistelee yhden tai useamman aiheen (ongelman), joita he haluavat käsitellä juurisyyanalyysi-sessiossa. Kandidaatintyön tekijä on ohjaamassa sessiota ja työn valvoja seuraa tapahtumia sivusta. Työkaluna käytetään Aalto-yliopistolla kehitettyä ARCA-tool -web-sovellusta mahdollistamaan hajautetun tiimin yhteisen analyysin. Sessio videoidaan ruudunkaappaus-sovelluksella. 

Juurisyyanalyysi-session jälkeen osallistuneilta kerätään palaute ennalta tehdyn palaute-lomakkeen avulla. Lisäksi kerätään suullinen palautte. Kysymykset käsittelevät tiiimin suorittaman juurisyyanalyysin ja siihen käytetyn työkalun helppokäyttöisyyttä ja tehokkuutta. Kenttätutkimuksen analysointi tapahtuu tämän kerätyn aineiston perusteella. Tutkimuksessa käytetty juurisyyanalyysimenetelmä on pelkistetty versio ARCA-menetelmästä \citep{Lehtinen2011}.

Kandidaatintyössä esitellään ensin lyhyesti työn ymmärtämiseen tarvittava teoreettinen tausta, eli määrittellään termit "ketterä retrospektiivi" ja "juurisyyanalyysi". Tämän jälkeen kuvataan työssä käytettyjä menetelmiä, eli järjestelmällistä kirjallisuuskatsausta ja kenttätutkimuksen menetelmiä. Sitten tuodaan julki näillä menetelmillä saadut tulokset ja tehdään näistä tarvittavat johtopäätökset. Kandidaatintyön viimeinen kappale on yhteenveto.

\section{Teoreettinen tausta}
\subsection{Ketterä retrospektiivi}
Ketterissä ohjelmistokehitysmenetelmissä on määritelty, että on säännöllisin väliajoin pidetetävä reflektointi, jossa tiimi pohtii tapoja tulla tehokkaammaksi. Nämä kehitettyjen parannusten perusteella tiimi muuttaa toimintaansa \citep{AgileManifestoPrinciples}. Kevyt retrospektiivi sessio on yksi tapa toteuttaa tätä periaatetta. Projektin lopputuloksen kannalta retrospektiivejä on järkevää pitää lyhyin väliajoin. Tällöin tiimin kohtaamat ongelmat ja niihin liittyvät yksityiskohdat ovat tuoreessa muistissa ja parannusehdotukset voidaan ottaa suoraan käyttöön parantaen siten projektin lopputuloksen laatua. \citep{Cockburn2002}

\subsection{Juurisyyanalyysi}
Juurisyyanalyysi on rakenteellinen tapa tutkia ongelmia ja tunnistaa niiden aiheuttajia. Ideana on se, että korjaamalla syyn aiheuttavia ongelmia voidaan ehkäistä saman ongelman syntymistä uudestaan -- tai ainakin vähentää ongelman uudelleenesiintymisen todennäköisyyttä. \citep{Lehtinen2011} Juurisyyanalyysin avulla voidaan tutkia ongelmien lisäksi myös onnistumisten syitä. \citep{Bjornson2009} Juurisyylle on useita määritelmiä. Se voi tarkoittaa mitä tahansa ongelman aiheuttavaa syytä, syyketjun perimmäisintä syytä tai syyksi, johon johtoporras voi vaikuttaa. Juurisyyanalyysin tuloksia voidaan käyttää apuna prosessinkehityksessä. \citep{Lehtinen2011}

\section{Tutkimusmenetelmät}
\subsection{Systemaattinen kirjallisuuskatsaus}
Systemaattisessa kirjallisuuskatsauksessa (SLR, Systematic Literature Review) tehdään valitusta aiheesta kattava arviointi, jossa käytetään luotettavaa, tarkkaa ja toistettavisaa olevaa menetelmää. SLR on kirjallisuuskatsauksen muoto, eli siinä käydään läpi aiempia tutkimuksia, jotka ovat olennaisia omien tutkimuskysymysten valossa. Kerätyn kirjallisuuden pohjalta muodostetaan synteesi. \citep{Kitchenham2007}

SLR:n erityispiirteenä on se, että kirjallisuuden etsiminen, valikointi ja valitun kirjallisuuden analysointi pyritään tekemään toistettavasti ja puolueettomasti. Kitchenham perustelee SLR-menetelmää toteamalla, että kirjallisuuskatsaus, joka ei ole SLR:n tapaan perusteellinen ja tasapuolinen ei tarjoa paljoa tieteellistä arvoa. Sen tulokset ovat todennäköisemmin puolueetetomia, eli tutkijan kannasta riippumattomia. \citep{Kitchenham2007}

Systemaattinen kirjallisuuskatsaus koostuu kolmesta päävaiheesta: suunnittelu-, toteutus- ja raportointivaiheista. Suunnitteluvaiheessa tunnistetaan kirjallisuuskatsauksen tarve, määritellään tutkimuskysymykset, sekä protokolla katsauksen suorittamiselle. 

Toteutusvaiheessa tehdään aineistohakuja ja valitaan ennaltamääritellyin kriteerein tutkimukselle oleelliset artikkelit. Artikkelien laatua ja sitä kautta luotettavuutta arvioidaan. Tehdyistä hauista kirjataan kaikki kirjallisuuskatsauksen toistamiseen ja sen laadun arviointiin tarvittava tieto, kuten haussa käytetyt hakukoneet, hakusanat, löydettyjen tulosten määrä ja jopa tuloslista. Valitusta aineistosta kerätään tietoa talteen ja sen merkittävyyttä omalle tutkimukselle arvioidaan ennalta määritellyin kriteerein. Kerätyn ja merkittäväksi valitun tiedon perusteella muodostetaan synteesi. Raportointivaiheessa kirjoitetaan tulokset ylös ja arvioidaan tuloksia.

Tähän kandidaatintyöhön valittiin Kitchenhamin määrittelemä systemaattinen kirjallisuuskatsaus, jotta kerätty aineistolista ja sen pohjalta tehty synteesi olisi tieteellisesti merkittävämpää, sekä mahdollisesti myös muulle tutkimukselle käyttökelpoista.

\subsection{Kenttätutkimus}
Kenttätutkimus suoritetaan keskikokoisessa ohjelmistoyrityksessä, jonka ketterälle ohjelmistokehitystiimille kandidaatintyön kirjoittaja pitää retrospektiivin. Kyseessä on noin kymmenen jäsenen kokoinen eri maiden välillä toimiva hajautettu tiimi. Retrospektiivissä tiimi pohtii edellisessä sprintissä haasteita aiheuttaneita ongelmia höyhenenkevyttä juurisyyanalyysi-menetelmää käyttäen. Retrospektiivin kommunikaatio tapahtuu videoneuvottelutyökalua käyttäen. Juurisyyanalyysissä käytetään Aalto-yliopistolla kehitettyä ARCA-tool -juurysyyanalyysityökalun eri versioita. ARCA-tool on Internetin ylitse toimiva juurisyyanalyysityökalu, joka mahdollistaa kollaboratiivisen ja reaaliaikaisen juurisyyanalyysin tekemisen. Jokainen voi osallistua analyysin tekemiseen omalta työpisteeltään pelkää Internet-selainta käyttäen.

Itse juurisyyanalyysimenetelmän on täytettävä tietyt kriteerit, jotta se soveltuisi tiimin retrospektiiviin. Ensinnäkin tiimi on asettanut retrospektiivin aikaikkunaksi yhden tunnin. Tiimin pitää siis saada valittua juurisyyanalyysimenetelmää käyttämällä konkreettisia tuloksia tuon tunnin jälkeen. Tavoitteena on ymmärtää tiimin valitsemat ongelmat ja niiden aiheuttajat syvällisemmin, sekä muodostaa konkreettisia toimia tiimin seuraavan sprintin backlogille näiden ongelmien ehkäisemiseksi. Lisäksi menetelmän pitää olla helposti omaksuttava siten, ettei retrospektiiviin osallistuville tarvitse pitää erillistä koulutusta menetelmästä, vaan se voidaan kouluttaa heille muutamalla lauseella ennen aloittamista.

Juurisyyanalyysi-sessiot taltioidaan ruudunkaappaus-sovelluksella, jolloin sisältyy sekä ARCA-tool:in näkymä koko analyysin ajan, sekä kommunikointiin käytetty videoneuvottelukuva. Taltioinnin tarkoituksena on se, ettei kandidaatintyön kirjoittalta jää huomaamatta yksityiskohtia session kulusta, osallistujien tunnetiloista ja reaktioista tai syyverkon muodostumisesta.

Sessioiden jälkeen kandidaatintyön tekijä kerää osallistujilta palautteen, joka koostuu sekä kyselylomakkeesta, että vapaammasta suullisesta tai pikaviestimen avulla kerätystä palautteesta. Molemmissa esitetyt kysymykset käsittelevät sitä, miten tehokkaaksi ja helppokäyttöiseksi osallistujat kokivat käytetyn juurisyyanalyysimenetelmän ja ARCA-tool:in verrattuna aiempiin prosessinkehitysmenetelmiinsä. Kyselylomake on Google Forms -työkalulla tehty Internet-lomake, jossa on yhteensä 26 kysymystä. 

\subsubsection{Ensimmäinen case}
Ensimmäisen retrospektiivin juurisyyanalyysimenetelmäksi on valittu ARCA-menetelmään perustuva, erittäin yksinkertainen ja pelkistetty,  höyhenenkevyt menetelmä, jonka kuvaan seuraavaksi.

Retrospektiivin lähtöasetelmana on se, että kullakin tiimin jäsenellä on Internet-selaimessaan auki retrospektiiviä varten luotu uusi analyysi ARCA-tool:issa. Tiimi on etukäteen valinnut ongelmat, joihin analyysissä halutaan pureutua.

Fasilitaattorin ohjauksessa tiimin jäsenet keräävät ongelmiin johtaneita syitä. Syiden ja niiden alisyiden kerääminen tehdään kahdessa iteraatioissa. Näissä viiden minuutin pituisissa iteraatioissa tiimin jäsenet lisäävät omatoimisesti ongelmiin ja niiden syihin johtaneita syitä ARCA-tool:issa näkyvään syyverkkoon. Kunkin iteraation jälkeen lisätyt syyt käydään läpi, jotta kaikki saavat yleiskuvan siitä, minkälaisista syistä syyverkko koostuu. Jokainen tiimin jäsen esittelee lisäämänsä syyt. Mikäli läpikäynnin yhteydessä esiintyy uusia syitä, ne lisätään verkkoon. Toisen iteraation jälkeen tiimin scrum master summaa koko syyverkon lyhyesti. 

Viimeinen askel on juurisyiden löytäminen. Fasilitaattori korostaa sitä, että juurisyyt, joita retrospektiivissä haetaan ovat ne syvimmät syyt, joihin tiimi voi vaikuttaa. Mielellään ne ovat sellaisia, joihin vaikuttamisen voi siirtää seuraavan sprintin backlogille. Juurisyiden löytäminen tapahtuu siten, että jokainen tiimin jäsen saa ehdottaa yhtä syytä "tykkäämällä" siitä ARCA-tool:issa. Ne 1-3 syytä, jotka ovat saaneet eniten tykkäyksiä, valitaan juurisyiksi.

Ensimmäisessä case-retrospektiivissä käytetään ARCA-tool:in ensimmäistä versiota, joka kehitettiin 2011-2012 lukuvuonna Aalto-yliopiston Ohjelmistokehitysprojekti-kurssilla (T-76.4115). Siinä on kaikki olennaisimmat juurisyyanalyysin tekemiseen tarvittavat toiminnot, jotka sisältävät muun muassa syyverkon kollaboratiivisen rakentamisen, tiettyjen syiden painottamisen tykkäys-toiminnolla, sekä korjaavien ideoiden lisäämisen.

\subsubsection{Toinen case}
Mikäli toinen RCA-sessio saadaan järjestettyä case-yrityksen tiimin kanssa, käytetään siinä eri juurisyyanalyysimenetelmää ja eri versiota ARCA-tool:ista. Menetelmäksi valitaan systemaattisen kirjallisuuskatsauksen tuloksena syntynyttä synteesiä. Voi olla, että synteesiin täytyy tehdä muutoksia, jotta se sopii tiimin retrospektiiville asettamiin raameihin. Retrospektiivissä käytetään ARCA-toolin uusinta versiota, joka kehitettiin 2012-2013 lukuvuonna Aalto-yliopiston Ohjelmistokehitysprojekti-kurssilla. Siinä on parannettu edellisen version käytettävyyttä, sekä lisätty uutena ominaisuutena mahdollisuus saada tehdystä analyysistä yksinkertaistetun yleiskuvan tarjoava näkymä.

\section{Tulokset}
\subsection{Systemaattinen kirjallisuuskatsaus}
\subsection{Kenttätutkimus}


\section{Pohdinta}

\section{Yhteenveto}

%\clearpage                     % luku loppuu, loput kelluvat tänne, sivunv.

%\input{luku2}                  % tässä tyylissä ei sivunvaihtoja lukujen
%\input{luku3}                  %   välillä. Toiset ohjaajat haluavat 
%\input{luku4}                  %   sivunvaihdot.

\label{pages:text}
\clearpage                     % luku loppuu, loput kelluvat tänne, sivunvaihto
%\newpage                       % ellei ylempi tehoa, pakota lähdeluettelo 
                               % alkamaan uudelta sivulta

% -------------- Lähdeluettelo / reference list -----------------------
%
% Lähdeluettelo alkaa aina omalta sivultaan; pakota lähteet alkamaan
% joko \clearpage tai \newpage
%
%
% Muista, että saat kirjallisuusluettelon vasta
%  kun olet kääntänyt ja kaulinnut "latex, bibtex, latex, latex"
%  (ellet käytä Makefilea ja "make")

% Viitetyylitiedosto aaltosci_t.bst; muokattu HY:n tktl-tyylistä.
\bibliographystyle{aaltosci_t}
% Katso myös tämän tiedoston yläosan "preamble" ja siellä \bibpunct.

% Muutetaan otsikko "Kirjallisuutta" -> "Lähteet"
\renewcommand{\refname}{Lähteet}  % article-tyyppisen
%\renewcommand{\bibname}{Lähteet}  % jos olisi book, report-tyyppinen

% Lisätään sisällysluetteloon
\addcontentsline{toc}{section}{\refname}  % article
%\addcontentsline{toc}{chapter}{\bibname}  % book, report

% Määritä kaikki bib-tiedostot
\bibliography{lahteet}
%\bibliography{thesis_sources,ietf_sources}

\label{pages:refs}
\clearpage         % erotetaan mahd. liitteet alkamaan uudelta sivulta

% -------------- Liitteet / Appendices --------------------------------
%
% Liitteitä ei yleensä tarvita. Kommentoi tällöin seuraavat
% rivit.

% Tiivistelmässä joskus matemaattisen kaavan tarkempi johtaminen, 
% haastattelurunko, kyselypohja, ylimääräisiä kuvia, lyhyitä 
% ohjelmakoodeja tai datatiedostoja.

\appendix

\section{Systemaattisen kirjallisuuskatsauksen hakutermien evoluutio}
\label{sec:hakutermi_evoluutio}


\begin{center}
\begin{longtable}{|p{1cm}|p{6cm}|p{2cm}|p{7cm}|}
\caption{Systemaattisen kirjallisuuskatsauksen hakutermien evoluutio}\label{haku_evoluutio_taulukko}\\ \hline

\textbf{\#} & \textbf{Hakutermi} & \textbf{Tulosten määrä} & \textbf{Arvio} \\
\endfirsthead
\multicolumn{4}{c}%
{\tablename\ \thetable\ -- \textit{Jatkoa edelliseltä sivulta}} \\
\hline
 \textbf{\#} & \textbf{Hakutermi} & \textbf{Tulosten määrä} & \textbf{Arvio} \\
\hline
\endhead
\hline \multicolumn{4}{r}{\textit{Jatkuu seuraavalla sivulla}} \\
\endfoot
\hline
\endlastfoot
\hline
 1 & TITLE-ABS-KEY("retrospective" AND "software engineering") & 103 & \begin{itemize} \item kaikkia hakusanoja ja rajauksia ei käytetty \item paljon epäolennaisia hakutuloksia esimerkiksi lääketieteen alalta \end{itemize} \\
\hline
 2 & TITLE-ABS-KEY("retrospective" AND "software engineering") AND DOCTYPE(ar) AND PUBYEAR > 1989 & 25 & \begin{itemize} \item kaikkia hakusanoja ei käytetty \item rajaukset ok  \item liian vähän tuloksia \end{itemize}\\
\hline
 3 & TITLE-ABS-KEY("postmortem analysis" AND "software engineering") AND DOCTYPE(ar) AND PUBYEAR > 1989 & 4 & \begin{itemize} \item kaikkia hakusanoja ei käytetty\item liian vähän tuloksia \end{itemize} \\
\hline
 4 & TITLE-ABS-KEY("post project review" AND "software engineering") & 0 & ~ \\
\hline
 5 & TITLE-ABS-KEY("post-project review" AND "software engineering") & 0 & ~ \\
\hline
 6 & TITLE-ABS-KEY(("retrospective" OR "postmortem analysis" OR "post-project review") AND "software engineering") & 114 & \begin{itemize} \item kaikkia rajauksia ei käytetty\item paljon epäolennaisia hakutuloksia esimerkiksi lääketieteen alalta \end{itemize} \\
\hline
 7 & TITLE-ABS-KEY(("retrospective" OR "postmortem analysis" OR "post-project review") AND "software engineering") AND DOCTYPE(ar) AND PUBYEAR > 1989 & 28 & \begin{itemize} \item kaikki hakusanat ja sovitut rajaukset käytetty\item liian vähän tuloksia \end{itemize} \\
\hline
 8 & TITLE-ABS-KEY("retrospective" AND "software") AND DOCTYPE(ar) AND SUBJAREA(comp OR engi OR busi) AND PUBYEAR > 1989 & 133 & \begin{itemize} \item tiukennettu rajausta kattamaan pelkästään tietotekniikan, insinööritieteet ja kauppatieteet\item muutettu hakusana “software engineering” yleisemmäksi “software”, jotta tuloksia saataisiin lisää\item iso osa tuloksista oli tutkimuksen kannalta epäolennaisia \end{itemize} \\
\hline
 9 & TITLE-ABS-KEY(("retrospective" OR "process improvement") AND "software") AND DOCTYPE(ar) AND SUBJAREA(comp OR engi OR busi) AND PUBYEAR > 1989 & 852 & \begin{itemize} \item kokeiltiin lisätä hakusana “process improvement”, joka löytyi mm. Lehtisen ja Bjornsonin RCA-aiheisista artikkeleista\item liikaa ja epäolennaisia tuloksia \end{itemize} \\
\hline
 10 & TITLE-ABS-KEY(("retrospective" OR "process improvement") AND "software engineering") AND DOCTYPE(ar) AND SUBJAREA(comp OR engi OR busi) AND PUBYEAR > 1989 & 390 & \begin{itemize} \item tarkennettiin hakusanaa “software” takaisin “software engineering” hakusanaksi\item kokeiltiin hakusanaa “process improvement”, joka löytyi mm. Lehtisen ja Bjornsonin RCA-aiheisista artikkeleista\item liikaa ja epäolennaisia tuloksia \end{itemize} \\
\hline
 11 & TITLE-ABS-KEY(("retrospective" OR "postmortem analysis" OR "post-project review" OR "process improvement") AND "software engineering") AND DOCTYPE(ar) AND SUBJAREA(comp OR engi OR busi) AND PUBYEAR > 1989 & 393 & \begin{itemize} \item lisätty muut retrospektiivin kuvaamiseksi käytetyt termit hakusanoiksi\item edelleen liikaa hakutuloksia \end{itemize} \\
\hline
 12 & TITLE-ABS-KEY(("retrospective" OR "postmortem analysis" OR "post-project review" OR "process improvement") AND "software engineering") AND DOCTYPE(ar) AND SUBJAREA(comp) AND PUBYEAR > 1989 & 326 & \begin{itemize} \item tarkennettu aihealueen rajausta pelkään tietotekniikkaan\item liikaa hakutuloksia \end{itemize} \\
\hline
 13 & TITLE-ABS-KEY(("retrospective" OR "postmortem analysis" OR "post-project review" OR "process improvement") AND "software engineering") AND DOCTYPE(ar) AND SUBJAREA(comp) AND PUBYEAR > 1999 & 233 & \begin{itemize} \item tiukennettu aikarajausta vuoteen 2000\item liikaa hakutuloksia \end{itemize} \\
\hline
 14 & ALL("software engineering") AND TITLE-ABS-KEY("retrospective" OR "postmortem analysis" OR "post-project review") AND DOCTYPE(ar) AND SUBJAREA(comp) AND PUBYEAR > 1989 & 58 & \begin{itemize} \item laajennettiin hakua: hakusanaa “software engineering” ei enää haeta pelkästään otsikosta, abstraktista ja artikkelin avainsanoista, vaan kaikista mahdollisista kentistä.\item laajennettiin aikarajaa takaisin vuoteen 1990\item tulokset olennaisen näköisiä \item lupaavaa, mutta edelleen liian vähän tuloksia \end{itemize} \\
\hline
 15 & ALL("software engineering") AND TITLE-ABS-KEY("retrospective" OR "postmortem analysis" OR "pma" OR "post-mortem" OR "post mortem analysis" OR "post-project review" OR "post project review") AND DOCTYPE(ar) ANDSUBJAREA(comp) AND PUBYEAR > 1989 & 78 & \begin{itemize} \item tuloksia saatiin lisää sallimalla postmortem-analyysin eri kirjoitusasuja\item edelleen liian vähän tuloksia \end{itemize} \\
\hline
 16 & TITLE-ABS-KEY("retrospective" OR "postmortem analysis" OR "pma" OR "post-mortem" OR "post mortem analysis" OR "post-project review" OR "post project review") AND DOCTYPE(ar) AND SUBJAREA(comp) AND PUBYEAR > 1989 & 945 & \begin{itemize} \item kokeiltiin poistaa hakusana “software engineering”, mikäli se olisi rajoittanut tuloksia liikaa\item liikaa tuloksia \end{itemize} \\
\hline
 17 & ALL("agile") AND TITLE-ABS-KEY("retrospective" OR "postmortem analysis" OR "pma" OR "post-mortem" OR "post mortem analysis" OR "post-project review" OR "post project review") AND DOCTYPE(ar) AND SUBJAREA(comp) AND PUBYEAR > 1989 & 20 & \begin{itemize} \item korvattiin hakusana “software engineering” hakusanalla “agile”\item liian vähän tuloksia \end{itemize} \\
\hline
 18 & ALL("software development") AND TITLE-ABS-KEY("retrospective" OR "postmortem analysis" OR "pma" OR "post-mortem" OR "post mortem analysis" OR "post-project review" OR "post project review") AND DOCTYPE(ar) AND SUBJAREA(comp) AND PUBYEAR > 1989 & 51 & \begin{itemize} \item korvattiin hakusana “software engineering” hakusanalla “software development”\item liian vähän tuloksia \end{itemize} \\
\hline
 19 & ALL("software engineering") AND TITLE-ABS-KEY("retrospective" OR "postmortem analysis" OR "pma" OR "post-mortem" OR "post mortem analysis" OR "post-project review" OR "post project review" OR "root cause analysis" OR"rca" OR "defect cause analysis" OR "dca") AND DOCTYPE(ar) AND SUBJAREA(comp) AND PUBYEAR > 1989 & 108 & \begin{itemize} \item otettu mukaan juurisyyanalyysiin liittyviä hakusanoja\item tähän asti: haettu retrospektiivillä ja etsitty sieltä retro-menetelmiä, joissa on rca. nyt: haettu lisäksi juurisyyanalyysillä ja etsitty tästä hakujoukosta tapauksia, joissa rca:ta käytetään retron menetelmänä.\item kokeiltu siis lähestyä asiaa kahdelta suunnalta, retron suunnalta ja rca:n suunnalta\item muuten sama, kuin Haku 13, mutta RCA lisätty\item tuloksia sopivasti\item tulokset vaikuttavat hyviltä\item ohjaajan suosittelemat RCA-artikkelien kirjoittajat löytyivät (esim Bjornson, Card, …)\item valittiin lopulliseksi tulosjoukoksi \end{itemize} \\
\hline
\end{longtable}
\end{center}


%\newpage
%\section{Toinen esimerkkiliite}
%\label{sec:app2}
%
%Haastattelukysymykset: mitä, missä, milloin, kuka, miten.



\label{pages:appendices}

% ---------------------------------------------------------------------

\end{document}
