% Tiivistelmät tehdään viimeiseksi. 
%
% Avainsanojen lista pitää merkitä main.tex-tiedoston kohtaan \KEYWORDS.

\begin{fiabstract}

Ketterä tuotekehityksessä ohjelmistokehitystiimi pitää säännöllisin väliajoin retrospektiivin, jonka avulla se pyrkii parantamaan omaa työskentelyprosessiaan. Ketterää ohjelmistokehitystä kuvaavat metodologiat eivät kuitenkaan määrittele retrospektiivin toteutustapaa. Juurisyyanalyysi tarjoaa rakenteellisen tavan tutkia ongelmien aiheuttajia ja voi siten sopia hyvin retrospektiivin ongelmanratkaisuun. 

Tässä kandidaatintyössä pyritään löytämään systemaattisen kirjallisuuskatsauksen avulla vastaus siihen, minkälaisia menetelmiä aiemmassa kirjallisuudessa on esitetty juurisyyanalyysiä soveltaviin retrospektiiveihin. Menetelmät kuvataan, niitä vertaillaan keskenään ja niitä analysoidaan. Menetelmien perusteella kehitetään synteesi, joka kuvaa ketterään retrospektiiviin soveltuvan menetelmän.

Iso osa artikkeleista käytti olennaisten ongelmien ja onnistumisten tunnistamiseen KJ-menetelmää, kausaalianalyysiin fasilitoitua ryhmäkeskustelua käyttäen Ishikawan kalanruotodiagrammia ongelmien syy-seuraus-suhteiden esittämiseen. Uusimmat artikkelit käyttivät kausaalianalyysiin KJ-menetelmän tapaista osallistavaa menetelmää ja suunnattua verkkoa ongelmien syy-seuraus-esitystapana. Alle puolet kuvatuista menetelmistä sisälsi kehitysehdotuksien kehittämisvaiheen.

Muodostettu synteesi on seuraava. Kausaalianalyysin syöte kehitetään käyttäen KJ-menetelmää ja olennaisten asioiden valintaan käsiäänestystä. Kausaalianalyysissä muodostetaan KJ-menetelmää käyttäen suunnattu verkko visualisoimaan syy-seuraus-suhteet. Juurisyyt valitaan käsiäänestyksellä. Brainstoring-menetelmän avulla muodostetaan parannusideataulukko, josta toteutettavat ideat äänestetään käsiäänestyksellä.

\end{fiabstract}
